% t tex proofs_in_analysis.tex  &&  sheaf proofs_in_analysis.dvi
% t tex proofs_in_analysis.tex  &&  t dvipdfm -p letter -m 0.956 -x 0.95in -y 0.95in proofs_in_analysis.dvi  &&  mupdf-gl proofs_in_analysis.pdf

% mupdf-gl  proofs_in_analysis.pdf
% mupdf-x11 proofs_in_analysis.pdf
% xpdf      proofs_in_analysis.pdf

% `aka` should go *without* parentheses!

% TeXBook p71
% TeX recognizes several kinds of infinity, some of which are “more infinite” than others.
% You can say both \vfil and \vfill; the second is stronger than the first.
% In other words, if no other infinite stretchability is present, \vfil will expand to fill the remaining space;
% but, if both \vfil and \vfill are present simultaneously, the \vfill effectively prevents \vfil from stretching.
% You can think of it as if \vfil has one mile of stretchability, while \vfill has a trillion miles.




% ------------------------------------------------------------------------------------------------------------------------------
% ------------------------------------------------------------------------------------------------------------------------------
% ------------------------------------------------------------------------------------------------------------------------------
% ------------------------------------------------------------------------------------------------------------------------------
\nonstopmode
\parindent=0pt

\magnification1000  % Global font size! Activate mag@960 when rendering to PDF!  BUG in TeX/DVI? This needs to go BEFORE hsize/vsize, for some reason!
\hsize  = 8.7in  % Dimension of the TEXT BLOCK! Default vals are \hsize=6.5in, \vsize=8.9in
\vsize  =11.0in  % Dimension of the TEXT BLOCK! Default vals are \hsize=6.5in, \vsize=8.9in
\hoffset=-0.9in  % Position  of the TEXT BLOCK! Default vals are 1 inch in from the left (\hoffset=0pt) and 1 inch from the top (\voffset=0pt). We need this for dvipdfm (whose output goes to chrome/xpdf) and xdvi!
\voffset=-0.9in  % Position  of the TEXT BLOCK! Default vals are 1 inch in from the left (\hoffset=0pt) and 1 inch from the top (\voffset=0pt). We need this for dvipdfm (whose output goes to chrome/xpdf) and xdvi!
\def\makeheadline{\vbox to 0pt{\vskip-22.5pt \line{\vbox to8.5pt{}\the\headline}\vss}\nointerlineskip}  % Set the spacing between the HEADER and the TEXT BLOCK!
\def\makefootline{\baselineskip24pt\lineskiplimit0pt\line{\the\footline}}                               % Set the spacing between the FOOTER and the TEXT BLOCK!

% \magnification1000  % Global font size! Activate mag@960 when rendering to PDF!  BUG in TeX/DVI? This needs to go BEFORE hsize/vsize, for some reason!
% \hsize  = 8.3in  % Dimension of the TEXT BLOCK! Default vals are \hsize=6.5in, \vsize=8.9in
% \vsize  =10.5in  % Dimension of the TEXT BLOCK! Default vals are \hsize=6.5in, \vsize=8.9in
% \hoffset=-0.9in  % Position  of the TEXT BLOCK! Default vals are 1 inch in from the left (\hoffset=0pt) and 1 inch from the top (\voffset=0pt). We need this for dvipdfm (whose output goes to chrome/xpdf) and xdvi!
% \voffset=-0.9in  % Position  of the TEXT BLOCK! Default vals are 1 inch in from the left (\hoffset=0pt) and 1 inch from the top (\voffset=0pt). We need this for dvipdfm (whose output goes to chrome/xpdf) and xdvi!
% \def\makeheadline{\vbox to 0pt{\vskip-22.5pt \line{\vbox to8.5pt{}\the\headline}\vss}\nointerlineskip}  % Set the spacing between the HEADER and the TEXT BLOCK!
% \def\makefootline{\baselineskip24pt\lineskiplimit0pt\line{\the\footline}}                               % Set the spacing between the FOOTER and the TEXT BLOCK!

% \magnification1000  % Global font size! Activate mag@960 when rendering to PDF!  BUG in TeX/DVI? This needs to go BEFORE hsize/vsize, for some reason!
% \hsize  = 8.1in  % Dimension of the TEXT BLOCK! Default vals are \hsize=6.5in, \vsize=8.9in
% \vsize  =11.5in  % Dimension of the TEXT BLOCK! Default vals are \hsize=6.5in, \vsize=8.9in
% \hoffset=-0.9in  % Position  of the TEXT BLOCK! Default vals are 1 inch in from the left (\hoffset=0pt) and 1 inch from the top (\voffset=0pt). We need this for dvipdfm (whose output goes to chrome/xpdf) and xdvi!
% \voffset=-0.9in  % Position  of the TEXT BLOCK! Default vals are 1 inch in from the left (\hoffset=0pt) and 1 inch from the top (\voffset=0pt). We need this for dvipdfm (whose output goes to chrome/xpdf) and xdvi!
% \def\makeheadline{\vbox to -5pt{\vskip-22.5pt \line{\vbox to8.5pt{}\the\headline}\vss}\nointerlineskip}  % Set the spacing between the HEADER and the TEXT BLOCK!
% \def\makefootline{\baselineskip10pt\lineskiplimit0pt\line{\the\footline}}                                % Set the spacing between the FOOTER and the TEXT BLOCK!

\special{background rgb 0.0313 0.0313 0.0313}  % Background color!
\special{color push rgb 1.0000 1.0000 1.0000}  % Foreground color! IF we don't PUSH the foreground color, THEN the color functions (eg. `\red{}`) don't work!

% -----------------------------------------------------------------------------------------------------------------------------#
\font\tensc=cmcsc10 at 10pt % small-caps  {\sc *}  % NOTE! `\tensc` is not a TeX command!
\def\sc {\tensc}  % Define a new {\sc *} command, much like {\bf *} and {\it *}. DON'T put spaces!!

% Load bigger fonts!
\font\titlefont     =cmb10 at 50pt
\font\subtitlefont  =cmr10 at 20pt
\font\chapterfont   =cmssbx10 at 13pt
\font\sectionfont   =cmssbx10 at 12pt
\font\subsectionfont=cmssbx10 at 10pt

% -----------------------------------------------------------------------------------------------------------------------------#
% \def\l    #1{{\special{color push rgb 1.0000 1.0000 1.0000}#1\special{color pop}}}  % WARN! \l is TAKEN BY TEX!!
% \def\d    #1{{\special{color push rgb 0.0313 0.0313 0.0313}#1\special{color pop}}}  % WARN! \d is TAKEN BY TEX!!
\def\r    #1{{\special{color push rgb 1.0000 0.4000 0.4000}#1\special{color pop}}}
\def\g    #1{{\special{color push rgb 0.3000 0.7000 0.6000}#1\special{color pop}}}
\def\b    #1{{\special{color push rgb 0.0000 0.6000 1.0000}#1\special{color pop}}}  % WARN! \b is TAKEN BY TEX!!
\def\p    #1{{\special{color push rgb 0.5000 0.4000 0.9000}#1\special{color pop}}}
\def\a    #1{{\special{color push rgb 1.0000 0.2000 0.4000}#1\special{color pop}}}
\def\y    #1{{\special{color push rgb 1.0000 1.0000 0.4000}#1\special{color pop}}}  % 0.900 0.900 0.500
\def\pk   #1{{\special{color push rgb 1.0000 0.7000 0.7000}#1\special{color pop}}}
% WARN! \u is TAKEN BY TEX!!
% WARN! \v is TAKEN BY TEX!!
% WARN! \t is TAKEN BY TEX!!
% WARN! \c is TAKEN BY TEX!!

% \def\lbf  #1{{\special{color push rgb 1.0000 1.0000 1.0000}{\bf #1}\special{color pop}}}
% \def\dbf  #1{{\special{color push rgb 0.0313 0.0313 0.0313}{\bf #1}\special{color pop}}}
\def\rbf  #1{{\special{color push rgb 1.0000 0.4000 0.4000}{\bf #1}\special{color pop}}}
\def\gbf  #1{{\special{color push rgb 0.3000 0.7000 0.6000}{\bf #1}\special{color pop}}}
\def\bbf  #1{{\special{color push rgb 0.0000 0.6000 1.0000}{\bf #1}\special{color pop}}}
\def\pbf  #1{{\special{color push rgb 0.5000 0.4000 0.9000}{\bf #1}\special{color pop}}}
\def\abf  #1{{\special{color push rgb 1.0000 0.2000 0.4000}{\bf #1}\special{color pop}}}
\def\ybf  #1{{\special{color push rgb 1.0000 1.0000 0.4000}{\bf #1}\special{color pop}}}  % 0.900 0.900 0.500
\def\pkbf #1{{\special{color push rgb 1.0000 0.7000 0.7000}{\bf #1}\special{color pop}}}

% \def\lsc  #1{{\special{color push rgb 1.0000 1.0000 1.0000}{\sc #1}\special{color pop}}}
% \def\dsc  #1{{\special{color push rgb 0.0313 0.0313 0.0313}{\sc #1}\special{color pop}}}
\def\rsc  #1{{\special{color push rgb 1.0000 0.4000 0.4000}{\sc #1}\special{color pop}}}
\def\gsc  #1{{\special{color push rgb 0.3000 0.7000 0.6000}{\sc #1}\special{color pop}}}
\def\bsc  #1{{\special{color push rgb 0.0000 0.6000 1.0000}{\sc #1}\special{color pop}}}
\def\psc  #1{{\special{color push rgb 0.5000 0.4000 0.9000}{\sc #1}\special{color pop}}}
\def\asc  #1{{\special{color push rgb 1.0000 0.2000 0.4000}{\sc #1}\special{color pop}}}
\def\ysc  #1{{\special{color push rgb 1.0000 1.0000 0.4000}{\sc #1}\special{color pop}}}  % 0.900 0.900 0.500
\def\pksc #1{{\special{color push rgb 1.0000 0.7000 0.7000}{\sc #1}\special{color pop}}}

% -----------------------------------------------------------------------------------------------------------------------------#
\def\N {{\bf N}}
\def\Z {{\bf Z}}
\def\Q {{\bf Q}}
\def\R {{\bf R}}
\def\C {{\bf C}}
% \def \H {{\bf H}}  % WARN! \H is TAKEN BY TEX!!
\def\S {{\bf S}}
\def\P {{\bf P}}

% -----------------------------------------------------------------------------------------------------------------------------#
\def\hs {\hskip10pt}  % hspace!
\def\vs {\vskip8pt}   % vspace!
\def\pagenumbers {\footline{\hss\tenrm\folio\hss}}

\def\center     #1{{\leftskip=0pt plus 1fil \rightskip=\leftskip \parfillskip=0pt #1 \par}}
\def\chapter    #1{{\vfill\break             \chapterfont \center{(\a{Chapter})   \hskip16pt #1} \vskip4pt}}
\def\section    #1{{\hskip8pt\hrule\vskip1pt \sectionfont         (\a{Section})   \hskip16pt #1  \vskip4pt}}
\def\subsection #1{{\hskip4pt                \subsectionfont      (\a{subsection})\hskip16pt #1  \vskip4pt}}

\def\definition  {\par\asc{DEFINITION}. \hskip4pt}
\def\theorem     {\par\asc{THEOREM}.    \hskip4pt}
\def\lemma       {\par\asc{LEMMA}.      \hskip4pt}
\def\proposition {\par\asc{PROPOSITION}.\hskip4pt}
\def\corollary   {\par\asc{COROLLARY}.  \hskip4pt}
\def\axiom       {\par\asc{AXIOM}.      \hskip4pt}
\def\proof       {\par\asc{proof}.      \hskip4pt}
\def\remark      {\par\asc{remark}.     \hskip4pt}
\def\example     {\par\asc{example}.    \hskip4pt}

% -----------------------------------------------------------------------------------------------------------------------------#
% NOTE! It's best to use as few custom commands as possible, particularly when they're simple!
\def\to     {\p\longrightarrow}
\def\from   {\p\longleftarrow}
\def\mapsto {\p\longmapsto}

\def\equals  {\hskip1pt \a{=}  \hskip1pt}
\def\pipe    {\hskip4pt \a{|}  \hskip4pt}

\def\lthen  {\Longrightarrow}                                  % Logical implication!
\def\liff   {\Longleftrightarrow}                              % Logical biconditional!
\def\lnthen {{\Relbar{\mkern-11mu}/{\mkern-11mu}\Rightarrow}}  % Logical non-implication!





% ------------------------------------------------------------------------------------------------------------------------------
% ------------------------------------------------------------------------------------------------------------------------------
% ------------------------------------------------------------------------------------------------------------------------------
% ------------------------------------------------------------------------------------------------------------------------------
\nopagenumbers

\topglue 0pt plus 2fill  % Vertically center text?
{\titlefont \center{Proofs in Analysis}}
{\subtitlefont \center{no step left behind}}

\topglue 0pt plus 4fill  % Vertically center text?
{\center{Diego Cortez}}
{\center{diego@mathisart.org}}
{\center{mathisart.org}}

\vfill\break
\pagenumbers




% ------------------------------------------------------------------------------------------------------------------------------
% ------------------------------------------------------------------------------------------------------------------------------
% ------------------------------------------------------------------------------------------------------------------------------
% ------------------------------------------------------------------------------------------------------------------------------
\chapter{Preface}

% ------------------------------------------------------------------------------------------------------------------------------
\vs\hrule\vskip1pt
\subsection{\bf What is mathematics?}

Mathematics is the \abf{exercise} of \abf{reason}. (And the study of reason itself.) \par
% Mathematics is the \a{\bf exercise} of reason, the exploration of the mathematical realm, the search for truth, the study of reason itself.

\vs
{\bf To do} mathematics is {\bf to exercise} our reason. \par
{\bf To exercise} our reason is {\bf to do} mathematics. \par

\vs
Mathematics is (also) the exploration of the \abf{math realm}.

% NOTE! I don't like including the next 2 lines here, because: 0) math is about TRUTH (not ideas), 1) it overcrowds on otherwise pristine and clear section
% \vs
% Mathematics is {\it not} about numbers. \par
% Mathematics is about {\bf ideas}.


% ------------------------------------------------------------------------------------------------------------------------------
\vs\hrule\vskip1pt
\subsection{\bf The cornerstone of mathematics}

\a{\bf Truth} is the cornerstone of mathematics. Without truth, there is no mathematics. \par

\vs
I like to think that the ``goal'' of mathematics is {\bf to find truths}. Or to find beautiful truths. Or something. \par
How do we go from one truth to the next? Via {\bf proof}. \par

\vs
\a{\bf Proof} is the lifeblood of mathematics, connecting truth to truth. \par  % conectando verdad a verdad

\vs
Come to think of it, maybe the ``goal'' of math is {\it not} to find truths, but {\bf to find proofs}...
since {\bf truth} is often inaccessible to math (in part due to incompleteness, nonconstructibility, uncertainty, undecidability, incomputability, ...). \par
\halign{#\hfil & #\hfil & #\hfil \cr
    \hs  {\it Not all that is {\bf true}   can be {\bf proven}},   & \hs ({\bf incompleteness}) \cr
    \hs  {\it not all that    {\bf exists} can be {\bf shown}}.    & \hs ({\bf nonconstructibility}) \cr
    % \hs not all that    unfolds      can be forecast, & ({\it uncertainty}) \cr
    % \hs not all those who wander are lost. \cr
    % \hs not all that we exists can be reasoned, \cr
    % \hs not all those who wander are lost. \cr
    % \hs not everything that {\it is true} can {\it be proven} to be {\bf true},  \cr
    % \hs not everything that {\it exists}  can {\it be proven} to {\bf exist}, \cr
}

% If you believe that {\bf math is about truth}, then you also believe that {\bf math is about proofs}. This book is about proofs. In {\bf analysis}.

% ------------------------------------------------------------------------------------------------------------------------------
% Understanding math ideas is hard. \par
% Explaining    math ideas in writing is harder. \par
% Understanding math ideas in writing is even harder. \par

% ------------------------------------------------------------------------------------------------------------------------------
\vs\hrule\vskip1pt
\subsection{\bf Two kinds of proof}

There are two kinds of proofs: {\bf formal proofs} and {\bf``social'' proofs}. \par

\vs
A {\bf formal proof} is a mechanical {\it tree} of (logical) {\bf sentences}. \par % Each {\bf sentence} is an {\bf axiom}, a {\bf hypothesis}, or a {\bf theorem}. \par
The {\it nodes} of the tree (ie. the sentences) are connected by {\bf deduction}. \par
The {\it root} of the tree is the sentence that we're proving. \par
Formal proofs are {\bf rigorous}. \par

\vs
A {\bf ``social'' proof} is a flabby argument for why a (logical) sentence {\it may} be {\bf true}. \par
``Social'' proofs give us a {\bf rough idea} of why a sentence {\it may} be {\bf true}. \par
``Social'' proofs {\it rarely} give us a {\bf good idea} in practice, since most of them {\bf skip lots of steps} (or worse: they leave them as ``{\it exercise}''). \par
``Social'' proofs are what we find in most textbooks (like this one). \par
``Social'' proofs are {\bf not rigorous}, by their vagueness and incompleteness. \par

\vs
Formal     proofs are the {\bf machine code}     of mathematics. \par
``Social'' proofs are the {\bf natural language} of mathematics. \par

% ------------------------------------------------------------------------------------------------------------------------------
\vs\hrule\vskip1pt
\subsection{\bf Skipping steps is evil}

There's {\it exactly one} trivial thing in math: {\bf skipping steps}. \par

\vs
It's easy to ``prove'' something when we {\bf skip steps}. For example, \par

\vs
\theorem The Riemann hypothesis. \par
\proof   Exercise. \par

\vs
A proof that skips steps is {\bf no proof at all}.
Just as the mathematics community shouldn't accept proofs with holes, a math student should {\it never} accept a proof with holes.
% (In particular, a math student should never {\it pay} for a textbook whose proofs have holes or whose exercises have no solutions.
% Heck, if my doctor told me go to cure myself, I'd realize he's a sham and pay that sucker no dime.)
\par

\vs
Yet, my experience is that proofs in textbooks are often full of holes: \par
  \hs stuff that is assumed, \par
  \hs stuff that is ambiguous, \par
  \hs stuff that is unclear, \par
  \hs stuff that is left to the reader, \par
  \hs stuff that is left as ``exercise'', \par
  \hs stuff that is left to context, \par
  \hs stuff that depends on stuff that hasn't been proved, \par
  \hs stuff that depends on itself (circularity), \par
  \hs stuff that is simply ignored. \par
All this makes for bad explanations. Good mathematics is pristine, precise. Bad explanations are bad mathematics.

\vs
It takes intelligence to communicate clearly. It's trivial to speak gobbledygook that others don't understand. \par

\vs
The {\bf burden of explanation} is on the teacher/writer, {\it not} on the student/reader! \par
A good doctor doesn't tell patients to ``treat themselves''. A student's job is {\it not} to ``convince himself''. \par
It's the {\it responsibility} of the teacher/writer to make himself understood. If he's not understood, then {\bf he has failed}. Badly. \par

% % ------------------------------------------------------------------------------------------------------------------------------
% \vs\hrule\vskip1pt
% \subsection{\bf Communication is not for everyone}

% It takes intelligence to communicate clearly. It's easy to speak gobbledygook that others don't understand. It's hard to be articulate.

% \vs
% If a student doesn't understand something, it's {\it because} the teacher has {\bf failed} to explain it. \par
% That's understandable. {\bf Explaining stuff} is not for everyone. \par

% \vs
% Explaining stuff is only for {\it smart} people. \par
% Explaining {\it hard} stuff... is only for the smartest. (I'm {\it not} smart.)

% ------------------------------------------------------------------------------------------------------------------------------
\vs\hrule\vskip1pt
\subsection{\bf Proofs: the good, the bad, and the awesome}

A {\bf good proof} is a proof where {\bf every step is ``easy'' to follow}, {\it and} {\bf no step is skipped}. \par

\vs
A {\bf bad proof} is a proof where {\bf some steps are hard to follow}, {\it or} {\bf some steps are skipped}. \par

\vs
The {\it hallmark} of a {\bf good proof} is that {\bf the reader doesn't need to do any work} to follow the proof. \par
In particular, the {\bf reader doesn't need to stop and think} about some step, {\it and} he {\bf doesn't need pen and paper to follow the proof} (the {\bf writer} has supplied all steps/calculations).

\vs
The {\it hallmark} of a {\bf bad proof} is that {\bf the reader needs to do some work} to follow the proof. \par
In particular, the {\bf reader needs to stop and think} about some step, {\it or} he {\bf needs pen and paper to follow the proof} (the {\bf writer} has skipped some steps/calculations).

\vs
An {\bf awesome proof} is a {\it good proof} that's also {\bf at the right level of abstraction}. \par
If the proof is too low-level, it'll be hard to {\it aggregate the details} into the high-level ideas of the proof. \par
If the proof is too abstract,  it'll be hard to {\it specialize the generalities} into the details of the proof. \par

\vs
Reading and understanding {\bf awesome proofs} is hard. \par
Reading and understanding {\bf good    proofs} is very hard. \par
Reading and understanding {\bf bad     proofs}... is near-impossible. \par

% ------------------------------------------------------------------------------------------------------------------------------
\vs\hrule\vskip1pt
\subsection{\bf The proofs in this book}

The proofs in this book are {\it not} {\bf good}, let alone {\bf awesome}. \par
All I promise is that I'm not actively trying to make them {\bf bad}. \par

% ------------------------------------------------------------------------------------------------------------------------------
% \vs\hrule\vskip1pt
% \subsection{\bf What the hay is this}

% This book isn't an enlightening, crystal-clear intro to analysis. \par
% It will not ``make you understand'' analysis. \par
% That's too hard for me. \par

% This book is just a compendium of proofs in analysis, as {\bf awesome} (in the sense above) as I could make them (which isn't saying much).

% The proofs in this book may still suck, though.

% \vs
% I don't know of any books that do it. \par
% Lara Alcock's {\it How To Think About Analysis} comes close. \par
% Daniel Velleman's {\it How To Prove It} is also very good. \par
% The proofs in those books are {\it good}. Sadly, there's {\it too few} of them! \par
% So I thought I'd compile detailed proofs of important/interesting results in analysis.




% ------------------------------------------------------------------------------------------------------------------------------
% ------------------------------------------------------------------------------------------------------------------------------
% ------------------------------------------------------------------------------------------------------------------------------
% ------------------------------------------------------------------------------------------------------------------------------
\chapter{No logic, no proof}

Mathematics is the exercise of reason. \par
When exercising our reason, we use {\bf logic}. \par
So, when doing math, we use logic. \par

% ------------------------------------------------------------------------------------------------------------------------------
\vs\hrule\vskip1pt
\axiom Creating new propositions from old propositions via logical connectives. \par
  \halign{#\hfil & #\hfil & #\hfil \cr
    \hs \hs For any propositions $P,Q \a\langle$\asc{not} $P$  &     & is a proposition$\a\rangle$, \cr
    \hs \hs For any propositions $P,Q \a\langle$$P$ \asc{and}  & $Q$ & is a proposition$\a\rangle$, \cr
    \hs \hs For any propositions $P,Q \a\langle$$P$ \asc{or}   & $Q$ & is a proposition$\a\rangle$, \cr
    \hs \hs For any propositions $P,Q \a\langle$$P$ \asc{then} & $Q$ & is a proposition$\a\rangle$, \cr
    \hs \hs For any propositions $P,Q \a\langle$$P$ \asc{iff}  & $Q$ & is a proposition$\a\rangle$, \cr
    % \hs\hs $\a\neg P$     & is a proposition, \cr
    % \hs\hs $P \p\land  Q$ & is a proposition, \cr
    % \hs\hs $P \p\lor   Q$ & is a proposition, \cr
    % \hs\hs $P \p\lthen Q$ & is a proposition, \cr
    % \hs\hs $P \p\liff  Q$ & is a proposition, \cr
    \hs $\a\rangle$ \cr
  }
\halign{#\hfil & #\hfil & #\hfil \cr
  \axiom The law of the & \abf{excluded middle}.  & For every proposition $P \a\langle$the proposition $P \p\lor  \a\neg P$ is \g{true}$\a\rangle$. \cr
  \axiom The law of     & \abf{noncontradiction}. & For every proposition $P \a\langle$the proposition $P \p\land \a\neg P$ is \r{false}$\a\rangle$. \cr
  \axiom The law of the & \abf{excluded middle}.  & $\a\forall P \a\langle$\asc{if} $P$ is proposition, \asc{then} $ P \p\lor \a\neg P \a\rangle$. \cr
  \axiom The law of     & \abf{noncontradiction}. & $\a\forall P \a\langle$\asc{if} $P$ is proposition, \asc{then} $ \a\neg\a(P \p\land \a\neg P\a) \a\rangle$. \cr
}

\vs
\theorem The law of the {\bf excluded middle} and the law of {\bf noncontradiction} are equivalent. \par

\proof \par
\a{C0}) \psc{Show}: the law of the {\bf excluded middle} is equivalent to the law of {\bf noncontradiction}. \par

\halign{#\hfil & #\hfil \cr
  $|$\hs \psc{Since}: & by definition, the law of the {\bf excluded middle}  is equivalent to $\a\forall P \a\langle P \p\lor \a\neg P \a\rangle$, \cr
  $|$\hs \psc{since}: & by definition, the law of     {\bf noncontradiction} is equivalent to $\a\forall P \a\langle \a\neg\a(P \p\land \a\neg P\a) \a\rangle$, \cr
  $|$\hs \psc{then}:  & by equivalence, \cr
  $|$\hs              & showing that the law of the {\bf excluded middle} is equivalent to the law of {\bf noncontradiction} \cr
  $|$\hs              & is equivalent to \cr
  $|$\hs              & showing that $\a\forall P \a\langle P \p\lor \a\neg P \a\rangle$ \asc{iff} $\a\forall P \a\langle \a\neg\a(P \p\land \a\neg P\a) \a\rangle$. \cr
}

$|$\hs \par
$|$\hs \a{C1}) \psc{Show}: $\a\forall P \a\langle P \p\lor \a\neg P \a\rangle$ \asc{iff} $\a\forall P \a\langle \a\neg\a(P \p\land \a\neg P\a) \a\rangle$. \par

$|$\hs$|$\hs \par
$|$\hs$|$\hs By $\a\forall$-hoisting, $\a\langle\a\forall P \a\langle P \p\lor \a\neg P \a\rangle$ \asc{iff} $\a\forall P \a\langle \a\neg\a(P \p\land \a\neg P\a) \a\rangle\a\rangle$ is equivalent to $\a\langle\a\forall P \a\langle P \p\lor \a\neg P$ \asc{iff} $\a\neg\a(P \p\land \a\neg P\a) \a\rangle\a\rangle$. (\a{C8}) \par

\halign{#\hfil & #\hfil & #\hfil & #\hfil & #\hfil \cr
  $|$\hs$|$\hs \cr
  $|$\hs$|$\hs \psc{Since}: & by \a{C8}),     &         & $\a\langle\a\forall P \a\langle P \p\lor \a\neg P \a\rangle$ \asc{iff} $\a\forall P \a\langle \a\neg\a(P \p\land \a\neg P\a) \a\rangle\a\rangle$ is equivalent to         & $\a\langle\a\forall P \a\langle P \p\lor \a\neg P$ \asc{iff} $\a\neg\a(P \p\land \a\neg P\a) \a\rangle\a\rangle$, \cr
  $|$\hs$|$\hs \psc{then}:  & by equivalence, & showing & $\a\langle\a\forall P \a\langle P \p\lor \a\neg P \a\rangle$ \asc{iff} $\a\forall P \a\langle \a\neg\a(P \p\land \a\neg P\a) \a\rangle\a\rangle$ is equivalent to showing & $\a\langle\a\forall P \a\langle P \p\lor \a\neg P$ \asc{iff} $\a\neg\a(P \p\land \a\neg P\a) \a\rangle\a\rangle$. \cr
}

$|$\hs$|$\hs \par
$|$\hs$|$\hs \a{C2}) \psc{Show}: $\a\langle\a\forall P \a\langle P \p\lor \a\neg P$ \asc{iff} $\a\neg\a(P \p\land \a\neg P\a) \a\rangle\a\rangle$. \par

\halign{#\hfil & #\hfil \cr
  $|$\hs$|$\hs$|$\hs \cr
  $|$\hs$|$\hs$|$\hs \a{H0}) \psc{Let}:  & $P$ is a proposition. \cr
  $|$\hs$|$\hs$|$\hs \a{C3}) \psc{Show}: & $P \p\lor \a\neg P$ \asc{iff} $\a\neg\a(P \p\land \a\neg P\a)$. \cr
}

\halign{#\hfil & #\hfil & #\hfil & #\hfil \cr
  $|$\hs$|$\hs$|$\hs$|$\hs \cr
  $|$\hs$|$\hs$|$\hs$|$\hs By the duality of conjuction and disjuction, & $\a\neg\a(P \p\land \a\neg P\a)$ & \asc{iff} $\a\neg P \p\lor \a\neg\a\neg P$. & (\a{C4}) \cr
  $|$\hs$|$\hs$|$\hs$|$\hs By \asc{not-not}-elimination,                & $\a\neg P \p\lor \a\neg\a\neg P$ & \asc{iff} $\a\neg P \p\lor P$. & (\a{C5}) \cr
  $|$\hs$|$\hs$|$\hs$|$\hs By the commutativity of disjunction,         & $\a\neg P \p\lor P$              & \asc{iff} $P \p\lor \a\neg P$. & (\a{C6}) \cr
  $|$\hs$|$\hs$|$\hs$|$\hs \cr
}
\halign{#\hfil & #\hfil & #\hfil & #\hfil \cr
  $|$\hs$|$\hs$|$\hs$|$\hs \psc{Since}: & by (\a{C4}),                        & $\a\neg\a(P \p\land \a\neg P\a)$ & \asc{iff} $\a\neg P \p\lor \a\neg\a\neg P$, \cr
  $|$\hs$|$\hs$|$\hs$|$\hs \psc{since}: & by (\a{C5}),                        & $\a\neg P \p\lor \a\neg\a\neg P$ & \asc{iff} $\a\neg P \p\lor P$, \cr
  $|$\hs$|$\hs$|$\hs$|$\hs \psc{since}: & by (\a{C6}),                        & $\a\neg P \p\lor P$              & \asc{iff} $P \p\lor \a\neg P$, \cr
  $|$\hs$|$\hs$|$\hs$|$\hs \psc{then}:  & by the transitivity of equivalence, & $\a\neg\a(P \p\land \a\neg P\a)$ & \asc{iff} $P \p\lor \a\neg P$. (\a{C7}) \cr
  $|$\hs$|$\hs$|$\hs$|$\hs \cr
}
\halign{#\hfil & #\hfil & #\hfil & #\hfil \cr
  $|$\hs$|$\hs$|$\hs$|$\hs \psc{Since}: & by (\a{C7}),                         & $\a\neg\a(P \p\land \a\neg P\a)$ & \asc{iff} $P \p\lor \a\neg P$, \cr
  $|$\hs$|$\hs$|$\hs$|$\hs \psc{then}:  & by the commutativity of equivalence, & $P \p\lor \a\neg P$              & \asc{iff} $\a\neg\a(P \p\land \a\neg P\a)$. (\a{C3}) \cr
  $|$\hs$|$\hs$|$\hs$|$\hs \cr
}
$|$\hs$|$\hs$|$\hs \psc{Shown} \a{C3}): $P \p\lor \a\neg P$ \asc{iff} $\a\neg\a(P \p\land \a\neg P\a)$. \par
$|$\hs$|$\hs$|$\hs \par

$|$\hs$|$\hs \psc{Shown} \a{C2}): $\a\langle\a\forall P \a\langle P \p\lor \a\neg P$ \asc{iff} $\a\neg\a(P \p\land \a\neg P\a) \a\rangle\a\rangle$. \par
$|$\hs$|$\hs \par

$|$\hs \psc{Shown} \a{C1}): $\a\forall P \a\langle P \p\lor \a\neg P \a\rangle$ \asc{iff} $\a\forall P \a\langle \a\neg\a(P \p\land \a\neg P\a) \a\rangle$. \par
$|$\hs \par

\psc{Shown} \a{C0}): the law of the {\bf excluded middle} is equivalent to the law of {\bf noncontradiction}. \par

% ------------------------------------------------------------------------------------------------------------------------------
\vs\hrule\vskip1pt

\definition The fundamental abstraction of $\a\exists$-syntax. The fundamental abstraction of $\a\forall$-syntax. \par
Let $x$ be a {\bf variable}. \par
Let $\g\varphi[x]$ be an {\bf open sentence} in $x$ (ie. $x$ is a free variable in $\g\varphi[x]$). \par
Let $X$ be a set. \par
\halign{#\hfil & #\hfil \cr
  \hs{\bf 0)}  & $\a\exists x \p\in X \a\langle$ $\g\varphi[x]$ $\a\rangle$ is \asc{defined} as $\a\exists x \a\langle$ $x \p\in X$ \asc{and} $\g\varphi[x]$ $\a\rangle$. \cr
  \hs{\bf 1)}  & $\a\forall x \p\in X \a\langle$ $\g\varphi[x]$ $\a\rangle$ is \asc{defined} as $\a\forall x \a\langle$ \asc{if} $x \p\in X$, \asc{then} $\g\varphi[x]$ $\a\rangle$. \cr
  \hs{\bf 0')} & $\a\exists x \p\in X \a\langle$ $\g\varphi[x]$ $\a\rangle$ is \asc{defined} as $\a\exists x \a\langle$ $x \p\in X$ $\a\land$ $\g\varphi[x]$ $\a\rangle$. \cr
  \hs{\bf 1')} & $\a\forall x \p\in X \a\langle$ $\g\varphi[x]$ $\a\rangle$ is \asc{defined} as $\a\forall x \a\langle$          $x \p\in X$  $\a\lthen$ $\g\varphi[x]$ $\a\rangle$. \cr
}

% ------------------------------------------------------------------------------------------------------------------------------
\vs\hrule\vskip1pt

\definition $\exists$-elimination, aka {\bf existential elimination}, aka existential instantiation. \par
TODO

\vs
\definition $\forall$-elimination, aka {\bf universal elimination}, aka universal instantiation. \par
TODO

% ------------------------------------------------------------------------------------------------------------------------------
\vs\hrule\vskip1pt
\theorem There exists a {\bf unicorn}. \par
Let $\a\langle$every lemon is yellow$\a\rangle$ be a proposition. \par
Let $\a\langle$there exists a unicorn$\a\rangle$ be a proposition. \par
  \hs{\bf 0)} \asc{if} $\a\langle$ $\a\langle$every lemon is yellow$\a\rangle$ \asc{and} \asc{not}$\a\langle$every lemon is yellow$\a\rangle$ $\a\rangle$, \par
  \hs\hs\asc{then} $\a\langle$there exists a unicorn$\a\rangle$. \par

% ----------------------------------------------------------------
\vs
\proof \par
\halign{#\hfil & #\hfil \cr
  \a{H0}) \psc{Let}:  & $\a\langle$every lemon is yellow$\a\rangle$ is a proposition. \cr
  \a{H1}) \psc{Let}:  & $\a\langle$there exists a unicorn$\a\rangle$ is a proposition. \cr
  \a{C0}) \psc{Show}: & \asc{if} $\a\langle$ $\a\langle$every lemon is yellow$\a\rangle$ \asc{and} \asc{not}$\a\langle$every lemon is yellow$\a\rangle$ $\a\rangle$, \asc{then} $\a\langle$there exists a unicorn$\a\rangle$. \cr
}

\halign{#\hfil & #\hfil \cr
  $|$\hs \cr
  $|$\hs \a{H2}) \psc{Let}:  & $\a\langle$every lemon is yellow$\a\rangle$. \cr
  $|$\hs \a{H3}) \psc{Let}:  & \asc{not}$\a\langle$every lemon is yellow$\a\rangle$. \cr
  $|$\hs \a{C1}) \psc{Show}: & $\a\langle$there exists a unicorn$\a\rangle$. \cr
}

\halign{#\hfil & #\hfil & #\hfil \cr
  $|$\hs$|$\hs \cr
  $|$\hs$|$\hs \psc{Since}: & by \a{H1}),               & $\a\langle$there exists a unicorn$\a\rangle$ is a proposition, \cr
  $|$\hs$|$\hs \psc{since}: & by \a{H2}),               & $\a\langle$every lemon is yellow$\a\rangle$, \cr
  $|$\hs$|$\hs \psc{then}:  & by \asc{or}-introduction, & $\a\langle$there exists a unicorn$\a\rangle$ \asc{or} $\a\langle$every lemon is yellow$\a\rangle$. (\a{C2}) \cr
  $|$\hs$|$\hs \cr
  $|$\hs$|$\hs \psc{Since}: & by \a{H1}),               & $\a\langle$there exists a unicorn$\a\rangle$ is a proposition, \cr
  $|$\hs$|$\hs \psc{since}: & by \a{H3}),               & \asc{not}$\a\langle$every lemon is yellow$\a\rangle$, \cr
  $|$\hs$|$\hs \psc{then}:  & by \asc{or}-introduction, & $\a\langle$there exists a unicorn$\a\rangle$ \asc{or} \asc{not}$\a\langle$every lemon is yellow$\a\rangle$. (\a{C3}) \cr
  $|$\hs$|$\hs \cr
  $|$\hs$|$\hs \psc{Since}: & by \a{C2}), & $\a\langle$there exists a unicorn$\a\rangle$ \asc{or} $\a\langle$every lemon is yellow$\a\rangle$, \cr
  $|$\hs$|$\hs \psc{since}: & by \a{C3}), & $\a\langle$there exists a unicorn$\a\rangle$ \asc{or} \asc{not}$\a\langle$every lemon is yellow$\a\rangle$, \cr
  $|$\hs$|$\hs \psc{then}:  & by ???,     & $\a\langle$there exists a unicorn$\a\rangle$. (\a{C1})\cr
  $|$\hs$|$\hs \cr
}

$|$\hs \psc{Shown}: \a{C1}) $\a\langle$there exists a unicorn$\a\rangle$. \par
$|$\hs \par
\psc{Shown}: \a{C0}) \asc{if} $\a\langle$ $\a\langle$every lemon is yellow$\a\rangle$ \asc{and} \asc{not}$\a\langle$every lemon is yellow$\a\rangle$ $\a\rangle$, \asc{then} $\a\langle$there exists a unicorn$\a\rangle$. \par

% ----------------------------------------------------------------
\vs
\proof \par
\halign{#\hfil & #\hfil \cr
  \a{H0}) \psc{Let}:  & $\a\langle$every lemon is yellow$\a\rangle$ is a proposition. \cr
  \a{H1}) \psc{Let}:  & $\a\langle$there exists a unicorn$\a\rangle$ is a proposition. \cr
  \a{C0}) \psc{Show}: & \asc{if} $\a\langle$ $\a\langle$every lemon is yellow$\a\rangle$ \asc{and} \asc{not}$\a\langle$every lemon is yellow$\a\rangle$ $\a\rangle$, \asc{then} $\a\langle$there exists a unicorn$\a\rangle$. \cr
}

\halign{#\hfil & #\hfil \cr
  $|$\hs \cr
  $|$\hs \a{H2}) \psc{Let}:  & $\a\langle$every lemon is yellow$\a\rangle$ is \g{true}. \cr
  $|$\hs \a{H3}) \psc{Let}:  & \asc{not}$\a\langle$every lemon is yellow$\a\rangle$ is \g{true}. \cr
  $|$\hs \a{C1}) \psc{Show}: & $\a\langle$there exists a unicorn$\a\rangle$ is \g{true}. \cr
}

\halign{#\hfil & #\hfil & #\hfil \cr
  $|$\hs$|$\hs \cr
  $|$\hs$|$\hs \psc{Since}: & by \a{H2}),               & $\a\langle$every lemon is yellow$\a\rangle$ is \g{true}, \cr
  $|$\hs$|$\hs \psc{since}: & by \a{H1}),               & $\a\langle$there exists a unicorn$\a\rangle$ is a proposition, \cr
  $|$\hs$|$\hs \psc{then}:  & by \asc{or}-introduction, & $\a\langle$every lemon is yellow \asc{or} there exists a unicorn$\a\rangle$ is \g{true}. (\a{C1}) \cr
  $|$\hs$|$\hs \cr
  $|$\hs$|$\hs \psc{Since}: & by \a{H3}),                   & \asc{not}$\a\langle$every lemon is yellow$\a\rangle$ is \g{true}, \cr
  $|$\hs$|$\hs \psc{then}:  & by negation,                  & $\a\langle$\asc{not not} every lemon is yellow$\a\rangle$ is \r{false}. (\a{C2}) \cr
  $|$\hs$|$\hs \psc{Since}: & by (\a{C2}),                  & $\a\langle$\asc{not not} every lemon is yellow$\a\rangle$ is \r{false}, \cr
  $|$\hs$|$\hs \psc{then}:  & by \asc{not-not}-elimination, & $\a\langle$every lemon is yellow$\a\rangle$ is \r{false}. (\a{C3}) \cr
  $|$\hs$|$\hs \cr
  $|$\hs$|$\hs \psc{Since}: & by \a{C1}),              & $\a\langle$every lemon is yellow \asc{or} there exists a unicorn$\a\rangle$ is \g{true}, \cr
  $|$\hs$|$\hs \psc{since}: & by \a{C3}),              & $\a\langle$every lemon is yellow$\a\rangle$ is \r{false}, \cr
  $|$\hs$|$\hs \psc{then}:  & by \asc{or}-elimination, & $\a\langle$there exists a unicorn$\a\rangle$ is \g{true}. (\a{C1}) \cr
  $|$\hs$|$\hs \cr
}

$|$\hs \psc{Shown}: \a{C1}) $\a\langle$there exists a unicorn$\a\rangle$ is \g{true}. \par
$|$\hs \par
\psc{Shown}: \a{C0}) \asc{if} $\a\langle$ $\a\langle$every lemon is yellow$\a\rangle$ \asc{and} \asc{not}$\a\langle$every lemon is yellow$\a\rangle$ $\a\rangle$, \asc{then} $\a\langle$there exists a unicorn$\a\rangle$. \par

% ----------------------------------------------------------------
\vs
\proof \par
\halign{#\hfil & #\hfil \cr
  \a{H0}) \psc{Let}:  & $\a\langle$every lemon is yellow$\a\rangle$ is a proposition. \cr
  \a{H1}) \psc{Let}:  & $\a\langle$there exists a unicorn$\a\rangle$ is a proposition. \cr
  \a{C0}) \psc{Show}: & \asc{if} $\a\langle$ $\a\langle$every lemon is yellow$\a\rangle$ \asc{and} \asc{not}$\a\langle$every lemon is yellow$\a\rangle$ $\a\rangle$, \asc{then} $\a\langle$there exists a unicorn$\a\rangle$. \cr
}

\halign{#\hfil & #\hfil & #\hfil \cr
  $|$\hs \cr
  $|$\hs \psc{Since}: & by \a{H0}),                           & $\a\langle$every lemon is yellow$\a\rangle$ is a proposition, \cr
  $|$\hs \psc{then}:  & by {\bf the law of noncontradiction}, & \asc{not} $\a\langle$ $\a\langle$every lemon is yellow$\a\rangle$ \asc{and} \asc{not}$\a\langle$every lemon is yellow$\a\rangle$ $\a\rangle$. (\a{C1}) \cr
  $|$\hs \cr
  $|$\hs \psc{Since}: & by \a{C1}),                           & \asc{not} $\a\langle$ $\a\langle$every lemon is yellow$\a\rangle$ \asc{and} \asc{not}$\a\langle$every lemon is yellow$\a\rangle$ $\a\rangle$, \cr
  $|$\hs \psc{then}:  & by {\bf the law of false antecedent}, & \asc{if} $\a\langle$ $\a\langle$every lemon is yellow$\a\rangle$ \asc{and} \asc{not}$\a\langle$every lemon is yellow$\a\rangle$ $\a\rangle$, \asc{then} $\a\langle$there exists a unicorn$\a\rangle$. \cr
  $|$\hs \cr
}

\psc{Shown}: \a{C0}) \asc{if} $\a\langle$ $\a\langle$every lemon is yellow$\a\rangle$ \asc{and} \asc{not}$\a\langle$every lemon is yellow$\a\rangle$ $\a\rangle$, \asc{then} $\a\langle$there exists a unicorn$\a\rangle$. \par

% ------------------------------------------------------------------------------------------------------------------------------
\vs\hrule\vskip1pt
\theorem The principle of explosion. \par
Let $P$ be a proposition. \par
Let $Q$ be a proposition. \par
  \hs{\bf 0)} \asc{if} $P$ is \g{true} \asc{and} $\neg P$ is \g{true}, \asc{then} $Q$ is \g{true}. \par

\vs
\proof \par
\psc{Let} $P$ be a proposition. \par
\psc{Let} $Q$ be a proposition. \par
\psc{Let} $P$      be \g{true}. \par
\psc{Let} $\neg P$ be \g{true}. \par
\psc{We show} that $Q$ is \g{true}. \par

  $|$\hs \par
  $|$\hs \psc{Since} $P$ is \g{true}, \psc{and} $Q$ is a proposition, \par
  $|$\hs \psc{then}, by \asc{or}-introduction, $P$ \asc{or} $Q$ is \g{true}. \par
  $|$\hs \par
  $|$\hs \psc{Since} $\neg P$ is \g{true}, \par
  $|$\hs \psc{then}, by negation, $\neg\neg P$ is \r{false}, \par
  $|$\hs \psc{Since} $\neg\neg P$ is \r{false}, \par
  $|$\hs \psc{then}, by $\neg\neg$-elimination, $P$ is \r{false}. \par
  $|$\hs \par
  $|$\hs \psc{Since} $P$ is \r{false}, \psc{and} $P$ \asc{or} $Q$ is \g{true}, \par
  $|$\hs \psc{then}, by \asc{or}-elimination, $Q$ is \g{true}. \par
  $|$\hs \par

\psc{This shows} that $Q$ is \g{true}. \par




% ------------------------------------------------------------------------------------------------------------------------------
% ------------------------------------------------------------------------------------------------------------------------------
% ------------------------------------------------------------------------------------------------------------------------------
% ------------------------------------------------------------------------------------------------------------------------------
\chapter{Sets and functions, a language for mathematics}



% ------------------------------------------------------------------------------------------------------------------------------
% ------------------------------------------------------------------------------------------------------------------------------
% ------------------------------------------------------------------------------------------------------------------------------
\section{Sets}

TODO



% ------------------------------------------------------------------------------------------------------------------------------
% ------------------------------------------------------------------------------------------------------------------------------
% ------------------------------------------------------------------------------------------------------------------------------
\section{Functions}

% ------------------------------------------------------------------------------------------------------------------------------
\vs
\definition {\bf Images} and {\bf preimages} of functions. \par
Let $X,Y$ be sets. \par
Let $f: X \to Y$ be a function. \par
Let $\g A \p\subseteq X$ be a subset of $X$. \par
Let $\b B \p\subseteq Y$ be a subset of $Y$. \par
\halign{#\hfil & #\hfil & #\hfil \cr
  \hs{\bf 0)} The \abf{image}    & of $\g A \p\subseteq X$, denoted $f_*[\g A]$, is the set $\a\{y \p\in Y \pipe \a\exists a_y \p\in \g A \a\langle$ $f: a_y \mapsto y$   $\a\rangle \a\}$. (Images    & \abf{can't be large}.) \cr
  \hs{\bf 1)} The \abf{preimage} & of $\b B \p\subseteq Y$, denoted $f^*[\b B]$, is the set $\a\{x \p\in X \pipe \a\exists b_x \p\in \b B \a\langle$ $f: x   \mapsto b_x$ $\a\rangle \a\}$. (Preimages & \abf{can't be small}.) \cr
}


% ------------------------------------------------------------------------------------------------------------------------------
% ------------------------------------------------------------------------------------------------------------------------------
\vs\hrule\vskip1pt
\theorem The fundamental lemma of functions. \par
Let $X,Y$ be sets. \par
Let $f: X \to Y$ be a function. \par
\halign{#\hfil & #\hfil \cr
  \hs{\bf 0)}  For every $\g A \p\subseteq X$ and $x \p\in X \a\langle$ \asc{if} $x \p\in \g A$,         & \asc{then} $f[x] \p\in f_*[\g A]$ $\a\rangle$. \cr
  \hs{\bf 1)}  For every $\g A \p\subseteq X$ and $x \p\in X \a\langle$ \asc{if} $f[x] \p\in f_*[\g A]$, & \asc{maybe not then} $x \p\in \g A$ $\a\rangle$. \cr
  % \hs{\bf 1')} For every $\g A \p\subseteq X$ and $x \p\in X \a\langle$ \asc{if} $f[x] \p\in f_*[\g A]$, & \asc{then} $x \p\in f^*[f_*[\g A]]$ $\a\rangle$. \cr
  \hs{\bf 2)}  For every $\b B \p\subseteq Y$ and $x \p\in X \a\langle$ \asc{if} $x \p\in f^*[\b B]$,    & \asc{then} $f[x] \p\in \b B$ $\a\rangle$. \cr
  \hs{\bf 3)}  For every $\b B \p\subseteq Y$ and $x \p\in X \a\langle$ \asc{if} $f[x] \p\in \b B$,      & \asc{then} $x \p\in f^*[\b B]$ $\a\rangle$. \cr
}
\halign{#\hfil & #\hfil \cr
  \hs{\bf 4)} For every $A_0,A_1 \p\subseteq X \a\langle$ \asc{if} $A_0 \p\subseteq A_1$, \asc{then} $f_*[A_0] \p\subseteq f_*[A_1]$ $\a\rangle$. (Images    & {\bf preserve subsets}.) \cr
  \hs{\bf 5)} For every $B_0,B_1 \p\subseteq Y \a\langle$ \asc{if} $B_0 \p\subseteq B_1$, \asc{then} $f^*[B_0] \p\subseteq f^*[B_1]$ $\a\rangle$. (Preimages & {\bf preserve subsets}.) \cr
}
  \hs{\bf 6)} For every $A \p\subseteq X$ and $B \p\subseteq Y \a\langle$ $A \p\subseteq f^*[B]$ \asc{iff} $f_*[A] \p\subseteq B$ $\a\rangle$. ({\bf Duality} of images and preimages.) \par

% ------------------------------------------------------------------------------------------------------------------------------
\ifnum 1=1
\vs
\asc{proof} of {\bf 0)}. \par
\psc{Let} $X,Y$ be sets. \par
\psc{Let} $f: X \to Y$ be a function. \par
\psc{Let} $\g A \p\subseteq Y$ be a subset of $X$. \par
\psc{We show} that for every $x \p\in X \a\langle$ \asc{if} $x \p\in \g A$, \asc{then} $f[x] \p\in f_*[\g A]$ $\a\rangle$. \par

  $|$\hs \par
  $|$\hs \psc{Let} $x \p\in X$ be an element of $X$. \par
  $|$\hs \psc{Let} $x \p\in \g A$ be an element of $\g A$. \par
  $|$\hs \psc{We show} that $f[x] \p\in f_*[\g A]$. \par

    $|$\hs$|$\hs \par
    $|$\hs$|$\hs By the image definition, $f_*[\g A]$ is the set $\a\{y \p\in Y \pipe \a\exists a_y \p\in \g A \a\langle$ $f: a_y \mapsto y$ $\a\rangle \a\}$. \par
    $|$\hs$|$\hs \psc{Since} \psc{We show} that $f[x] \p\in f_*[\g A]$, \par
    $|$\hs$|$\hs \psc{then}, by the image definition, \psc{We show} that there exists $a_y \p\in \g A$ so that $f: a_y \mapsto f[x]$. \par
    $|$\hs$|$\hs \psc{We show} that there exists $a_y \p\in \g A$ so that $f: a_y \mapsto f[x]$. \par

      $|$\hs$|$\hs$|$\hs \par
      $|$\hs$|$\hs$|$\hs \psc{Since} $x \p\in \g A$, \psc{and} $f: x \mapsto f[x]$, \psc{then} there exists $a_y \p\in \g A$ so that $f: a_y \mapsto f[x]$. \par
      $|$\hs$|$\hs$|$\hs \par

    $|$\hs$|$\hs \psc{This shows} that there exists $a_y \p\in \g A$ so that $f: a_y \mapsto f[x]$. \par
    $|$\hs$|$\hs \par

  $|$\hs \psc{This shows} that $f[x] \p\in f_*[\g A]$. \par
  $|$\hs \par

\psc{This shows} that for every $x \p\in X \a\langle$ \asc{if} $x \p\in \g A$, \asc{then} $f[x] \p\in f_*[\g A]$ $\a\rangle$. \par
\fi

% ------------------------------------------------------------------------------------------------------------------------------
\ifnum 1=1
\vs
\asc{proof} of {\bf 1)}. \par
TODO
\fi

% ------------------------------------------------------------------------------------------------------------------------------
\ifnum 1=1
\vs
\asc{proof} of {\bf 2)}. \par
\psc{Let} $X,Y$ be sets. \par
\psc{Let} $f: X \to Y$ be a function. \par
\psc{Let} $\b B \p\subseteq Y$ be a subset of $Y$. \par
\psc{We show} that for every $x \p\in X \a\langle$ \asc{if} $x \p\in f^*[\b B]$, \asc{then} $f[x] \p\in \b B$ $\a\rangle$. \par

  $|$\hs \par
  $|$\hs \psc{Let} $x \p\in X$ be an element of $X$. \par
  $|$\hs \psc{Let} $x \p\in f^*[\b B]$ be an element of the preimage $f^*[\b B]$. \par
  $|$\hs \psc{We show} that $f[x] \p\in \b B$. \par

    $|$\hs$|$\hs \par
    $|$\hs$|$\hs By the preimage definition, $f^*[\b B]$ is the set $\a\{x \p\in X \pipe \a\exists b_x \p\in \b B \a\langle$ $f: x \mapsto b_x$ $\a\rangle \a\}$. \par
    $|$\hs$|$\hs \psc{Since} $f:X \to Y$ is a function, \par
    $|$\hs$|$\hs \psc{then}, by the function definition, for every $x \p\in X$ and $y_0,y_1 \p\in Y \a\langle$ \asc{if} $f: x \mapsto y_0$ \asc{and} $f: x \mapsto y_1$, \asc{then} $y_0 \p= y_1$ $\a\rangle$. \par
    $|$\hs$|$\hs \psc{Since} $x \p\in f^*[\b B]$, \psc{then}, by the preimage definition, there exists $b_x  \p\in \b B$ so that $f: x \mapsto b_x$. \par
    $|$\hs$|$\hs \psc{Since} $x \p\in X$,         \psc{then}, by the function definition, there exists $f[x] \p\in Y$    so that $f: x \mapsto f[x]$. \par
    $|$\hs$|$\hs \psc{Since} $x \p\in X, b_x,f[x] \p\in Y$, \par
    $|$\hs$|$\hs \psc{and}   $f: x \mapsto b_x$, \par
    $|$\hs$|$\hs \psc{and}   $f: x \mapsto f[x]$, \par
    $|$\hs$|$\hs \psc{and}   for every $x \p\in X$ and $y_0,y_1 \p\in Y \a\langle$ \asc{if} $f: x \mapsto y_0$ \asc{and} $f: x \mapsto y_1$, \asc{then} $y_0 \p= y_1$ $\a\rangle$, \par
    $|$\hs$|$\hs \psc{then}, by setting $x \p\leftarrow x$ and $y_0 \p\leftarrow b_x$ and $y_1 \p\leftarrow f[x]$, $b_x$ is equal $f[x]$. \par
    $|$\hs$|$\hs \psc{Since} $b_x \p= f[x]$, \psc{and} $b_x \p\in \b B$, then, by replacement, $f[x] \p\in \b B$. \par
    $|$\hs$|$\hs \par

  $|$\hs \psc{This shows} that $f[x] \p\in \b B$.
  $|$\hs \par

\psc{This shows} that for every $x \p\in X \a\langle$ \asc{if} $x \p\in f^*[\b B]$, \asc{then} $f[x] \p\in \b B$ $\a\rangle$. \par
\fi

% ------------------------------------------------------------------------------------------------------------------------------
\ifnum 1=1
\vs
\asc{proof} of {\bf 3)}. \par
\psc{Let} $X,Y$ be sets. \par
\psc{Let} $f: X \to Y$ be a function. \par
\psc{Let} $\b B \p\subseteq Y$ be a subset of $Y$. \par
\psc{We show} that for for every $x \p\in X \a\langle$ \asc{if} $f[x] \p\in \b B$, \asc{then} $x \p\in f^*[\b B]$ $\a\rangle$. \par

  $|$\hs \par
  $|$\hs \psc{Let} $x \p\in X$ be an element of $X$. \par
  $|$\hs \psc{Let} $f[x] \p\in \b B$ be an element of $\b B$. \par
  $|$\hs \psc{We show} that $x \p\in f^*[\b B]$. \par

    $|$\hs$|$\hs \par
    $|$\hs$|$\hs By the preimage definition, $f^*[\b B]$ is the set $\a\{x \p\in X \pipe \a\exists b_x \p\in \b B \a\langle$ $f: x \mapsto b_x$ $\a\rangle \a\}$. \par
    $|$\hs$|$\hs \psc{Since} \psc{We show} that $x \p\in f^*[\b B]$, \par
    $|$\hs$|$\hs \psc{then}, by the preimage definition, \psc{We show} that there exists $b_x \p\in \b B$ so that $f: x \mapsto b_x$. \par
    $|$\hs$|$\hs \psc{We show} that there exists $b_x \p\in \b B$ so that $f: x \mapsto b_x$. \par

      $|$\hs$|$\hs$|$\hs \par
      $|$\hs$|$\hs$|$\hs \psc{Since} $f[x] \p\in \b B$, \psc{and} $f: x \mapsto f[x]$, \psc{then} there exists $b_x \p\in \b B$ so that $f: x \mapsto b_x$. \par
      $|$\hs$|$\hs$|$\hs \par

    $|$\hs$|$\hs \psc{This shows} that there exists $b_x \p\in \b B$ so that $f: x \mapsto b_x$. \par
    $|$\hs$|$\hs \par

  $|$\hs \psc{This shows} that $x \p\in f^*[\b B]$. \par
  $|$\hs \par

\psc{This shows} that for for every $x \p\in X \a\langle$ \asc{if} $f[x] \p\in \b B$, \asc{then} $x \p\in f^*[\b B]$ $\a\rangle$. \par
\fi

% ------------------------------------------------------------------------------------------------------------------------------
\ifnum 1=1
\vs
\asc{proof} of {\bf 4)}. \par
TODO
\fi

% ------------------------------------------------------------------------------------------------------------------------------
\ifnum 1=1
\vs
\asc{proof} of {\bf 5)}. \par
TODO
\fi

% ------------------------------------------------------------------------------------------------------------------------------
\ifnum 1=1
\vs
\asc{proof} of {\bf 6)}. \par
TODO
\fi


% ------------------------------------------------------------------------------------------------------------------------------
% ------------------------------------------------------------------------------------------------------------------------------
\vs\hrule\vskip1pt
\theorem The fundamental theorem of functions. \par
Let $X,Y$ be sets. \par
Let $f: X \to Y$ be a function. \par
\halign{#\hfil & #\hfil \cr
  \hs{\bf 0)} For every $\g A \p\subseteq X \a\langle$ $f^*[f_*[\g A]] \p\supseteq \g A$ $\a\rangle$. ({\bf Preimages} of images    {\bf can't be small}.) \cr
  \hs{\bf 1)} For every $\b B \p\subseteq Y \a\langle$ $f_*[f^*[\b B]] \p\subseteq \b B$ $\a\rangle$. ({\bf Images}    of preimages {\bf can't be large}.) \cr
}
\halign{#\hfil & #\hfil & #\hfil & #\hfil \cr
  \hs{\bf 2)} $f$ is {\bf injective}  & \asc{iff} for every $\g A \p\subseteq X \a\langle$ $f^*[f_*[\g A]] \p\subseteq \g A$ $\a\rangle$. ({\bf Preimages} & of injections  & are {\bf as small as possible}.) \cr
  \hs{\bf 3)} $f$ is {\bf surjective} & \asc{iff} for every $\b B \p\subseteq Y \a\langle$ $f_*[f^*[\b B]] \p\supseteq \b B$ $\a\rangle$. ({\bf Images}    & of surjections & are {\bf as large as possible}.) \cr
}

% ------------------------------------------------------------------------------------------------------------------------------
\ifnum 1=1
\vs
\asc{proof} of {\bf 0)}. \par
\psc{Let} $X,Y$ be sets. \par
\psc{Let} $f: X \to Y$ be a function. \par
\psc{Let} $\g A \p\subseteq X$ be a subset of $X$. \par
\psc{We show} that $f^*[f_*[\g A]] \p\supseteq \g A$. \par

$|$\hs \par
$|$\hs By the superset definition, $f^*[f_*[\g A]] \p\supseteq \g A$ is equivalent to $\a\langle$ for every $x \p\in \g A \a\langle$ $x \p\in f^*[f_*[\g A]]$ $\a\rangle$ $\a\rangle$. \par
$|$\hs \psc{Since} \psc{We show} that $f^*[f_*[\g A]] \p\supseteq \g A$, \par
$|$\hs \psc{and} $f^*[f_*[\g A]] \p\supseteq \g A$ is equivalent to $\a\langle$ for every $x \p\in \g A \a\langle$ $x \p\in f^*[f_*[\g A]]$ $\a\rangle$ $\a\rangle$, \par
$|$\hs \psc{then}, by replacement, \psc{We show} that for every $x \p\in \g A \a\langle$ $x \p\in f^*[f_*[\g A]]$ $\a\rangle$. \par
$|$\hs \psc{We show} that for every $x \p\in \g A \a\langle$ $x \p\in f^*[f_*[\g A]]$ $\a\rangle$. \par

$|$\hs$|$\hs \par
$|$\hs$|$\hs \psc{Let} $x \p\in \g A$ be an element of $\g A$. \par
$|$\hs$|$\hs \psc{We show} that $x \p\in f^*[f_*[\g A]]$. \par

$|$\hs$|$\hs$|$\hs \par
$|$\hs$|$\hs$|$\hs By the preimage definition, $f^*[f_*[\g A]]$ is the set $\a\{x \p\in X \pipe \a\exists b_x \p\in f_*[\g A] \a\langle$ $f: x   \mapsto b_x$ $\a\rangle \a\}$. \par
$|$\hs$|$\hs$|$\hs \psc{Since} \psc{We show} that $x \p\in f^*[f_*[\g A]]$, \par
$|$\hs$|$\hs$|$\hs \psc{then}, by the preimage definition, \psc{We show} that there exists $b_x \p\in f_*[\g A]$ so that $f: x \mapsto b_x$. \par
$|$\hs$|$\hs$|$\hs \psc{We show} that there exists $b_x \p\in f_*[\g A]$ so that $f: x \mapsto b_x$. \par

$|$\hs$|$\hs$|$\hs$|$\hs \par
$|$\hs$|$\hs$|$\hs$|$\hs By the fundamental abstraction of $\a\exists$-syntax, \par
$|$\hs$|$\hs$|$\hs$|$\hs the sentence $\a\langle$ there exists $b_x \p\in f_*[\g A]$ so that $f: x \mapsto b_x$ $\a\rangle$ is equivalent to the sentence $\a\exists b_x \a\langle$ $b_x \p\in f_*[\g A]$ \asc{and} $f: x \mapsto b_x$ $\a\rangle$. \par
$|$\hs$|$\hs$|$\hs$|$\hs \psc{Since} \psc{We show} that $\a\langle$ there exists $b_x \p\in f_*[\g A]$ so that $f: x \mapsto b_x$ $\a\rangle$, \par
$|$\hs$|$\hs$|$\hs$|$\hs \psc{and} the sentence $\a\langle$ there exists $b_x \p\in f_*[\g A]$ so that $f: x \mapsto b_x$ $\a\rangle$ is equivalent to the sentence $\a\exists b_x \a\langle$ $b_x \p\in f_*[\g A]$ \asc{and} $f: x \mapsto b_x$ $\a\rangle$, \par
$|$\hs$|$\hs$|$\hs$|$\hs \psc{then}, by replacement, \psc{We show} that $\a\exists b_x \a\langle$ $b_x \p\in f_*[\g A]$ \asc{and} $f: x \mapsto b_x$ $\a\rangle$. \par
$|$\hs$|$\hs$|$\hs$|$\hs \psc{We show} that $\a\exists b_x \a\langle$ $b_x \p\in f_*[\g A]$ \asc{and} $f: x \mapsto b_x$ $\a\rangle$. \par

$|$\hs$|$\hs$|$\hs$|$\hs$|$\hs \par
$|$\hs$|$\hs$|$\hs$|$\hs$|$\hs By the image definition, $f_*[\g A]$ is the set $\a\{y \p\in Y \pipe \a\exists a_y \p\in \g A \a\langle$ $f: a_y \mapsto y$ $\a\rangle \a\}$. \par
$|$\hs$|$\hs$|$\hs$|$\hs$|$\hs \psc{Since} \psc{must show} that $\a\exists b_x \a\langle$ $b_x \p\in f_*[\g A]$ \asc{and} $f: x \mapsto b_x$ $\a\rangle$, \par
$|$\hs$|$\hs$|$\hs$|$\hs$|$\hs \psc{then}, by the image definition, \psc{We show} that $\a\exists b_x \a\langle$ $\a\exists a_y \p\in A \a\langle f: a_y \mapsto b_x \a\rangle$ \asc{and} $f: x \mapsto b_x$ $\a\rangle$. \par
$|$\hs$|$\hs$|$\hs$|$\hs$|$\hs \psc{We show} that $\a\exists b_x \a\langle$ $\a\exists a_y \p\in A \a\langle f: a_y \mapsto b_x \a\rangle$ \asc{and} $f: x \mapsto b_x$ $\a\rangle$. \par

$|$\hs$|$\hs$|$\hs$|$\hs$|$\hs$|$\hs \par
$|$\hs$|$\hs$|$\hs$|$\hs$|$\hs$|$\hs By the fundamental abstraction of $\a\exists$-syntax, \par
$|$\hs$|$\hs$|$\hs$|$\hs$|$\hs$|$\hs the sentence $\a\exists a_y \p\in A \a\langle f: a_y \mapsto b_x \a\rangle$ is equivalent to the sentence $\a\exists a_y \a\langle a_y \p\in A$ \asc{and} $f: a_y \mapsto b_x \a\rangle$. \par
$|$\hs$|$\hs$|$\hs$|$\hs$|$\hs$|$\hs \psc{Since} \psc{We show} that $\a\exists b_x \a\langle$ $\a\exists a_y \p\in A \a\langle f: a_y \mapsto b_x \a\rangle$ \asc{and} $f: x \mapsto b_x$ $\a\rangle$, \par
$|$\hs$|$\hs$|$\hs$|$\hs$|$\hs$|$\hs \psc{and} the sentence $\a\exists a_y \p\in A \a\langle f: a_y \mapsto b_x \a\rangle$ is equivalent to the sentence $\a\exists a_y \a\langle a_y \p\in A$ \asc{and} $f: a_y \mapsto b_x \a\rangle$, \par
$|$\hs$|$\hs$|$\hs$|$\hs$|$\hs$|$\hs \psc{then}, by replacement, \psc{We show} that $\a\exists b_x \a\langle$ $\a\exists a_y \a\langle a_y \p\in A$ \asc{and} $f: a_y \mapsto b_x \a\rangle$ \asc{and} $f: x \mapsto b_x$ $\a\rangle$. \par
$|$\hs$|$\hs$|$\hs$|$\hs$|$\hs$|$\hs \psc{We show} that $\a\exists b_x \a\langle$ $\a\exists a_y \a\langle a_y \p\in A$ \asc{and} $f: a_y \mapsto b_x \a\rangle$ \asc{and} $f: x \mapsto b_x$ $\a\rangle$. \par

$|$\hs$|$\hs$|$\hs$|$\hs$|$\hs$|$\hs$|$\hs \par
$|$\hs$|$\hs$|$\hs$|$\hs$|$\hs$|$\hs$|$\hs \psc{Since} $x \p\in \g A$, \psc{and} $f: x \mapsto f[x]$, \par
$|$\hs$|$\hs$|$\hs$|$\hs$|$\hs$|$\hs$|$\hs \psc{then}, by setting $a_y \p\leftarrow x$ and $b_x \p\leftarrow f[x]$, we get that $\a\exists b_x \a\langle$ $\a\exists a_y \a\langle a_y \p\in A$ \asc{and} $f: a_y \mapsto b_x \a\rangle$ \asc{and} $f: x \mapsto b_x$ $\a\rangle$. \par
$|$\hs$|$\hs$|$\hs$|$\hs$|$\hs$|$\hs$|$\hs \par

$|$\hs$|$\hs$|$\hs$|$\hs$|$\hs$|$\hs \psc{This shows} that $\a\exists b_x \a\langle$ $\a\exists a_y \a\langle a_y \p\in A$ \asc{and} $f: a_y \mapsto b_x \a\rangle$ \asc{and} $f: x \mapsto b_x$ $\a\rangle$. \par
$|$\hs$|$\hs$|$\hs$|$\hs$|$\hs$|$\hs \par

$|$\hs$|$\hs$|$\hs$|$\hs$|$\hs \psc{This shows} that $\a\exists b_x \a\langle$ $\a\exists a_y \p\in A \a\langle f: a_y \mapsto b_x \a\rangle$ \asc{and} $f: x \mapsto b_x$ $\a\rangle$. \par
$|$\hs$|$\hs$|$\hs$|$\hs$|$\hs \par

$|$\hs$|$\hs$|$\hs$|$\hs \psc{This shows} that $\a\exists b_x \a\langle$ $b_x \p\in f_*[\g A]$ \asc{and} $f: x \mapsto b_x$ $\a\rangle$. \par
$|$\hs$|$\hs$|$\hs$|$\hs \par

$|$\hs$|$\hs$|$\hs \psc{This shows} that there exists $b_x \p\in f_*[\g A]$ so that $f: x \mapsto b_x$. \par
$|$\hs$|$\hs$|$\hs \par

$|$\hs$|$\hs \psc{This shows} that $x \p\in f^*[f_*[\g A]]$. \par
$|$\hs$|$\hs \par

$|$\hs \psc{This shows} that for every $x \p\in \g A \a\langle$ $x \p\in f^*[f_*[\g A]]$ $\a\rangle$. \par
$|$\hs \par

\psc{This shows} that $f^*[f_*[\g A]] \p\supseteq \g A$. \par
\fi

% ------------------------------------------------------------------------------------------------------------------------------
\ifnum 1=1
\vs
\asc{proof} of {\bf 1)}. \par
TODO
\fi

% ------------------------------------------------------------------------------------------------------------------------------
\ifnum 1=1
\vs
\asc{proof} of {\bf 2)}, only if. \par
\psc{Let} $X,Y$ be sets. \par
\psc{Let} $f: X \to Y$ be a function. \par
\psc{Let} $f: X \to Y$ be injective. \par
\psc{We show} that for every $\g A \p\subseteq X \a\langle$ $f^*[f_*[\g A]] \p\subseteq \g A$ $\a\rangle$.

  $|$\hs \par
  $|$\hs \psc{Let} $\g A \p\subseteq X$ be a subset of $X$. \par
  $|$\hs \psc{We show} that $f^*[f_*[\g A]] \p\subseteq \g A$. \par

    $|$\hs$|$\hs \par
    $|$\hs$|$\hs By the subset definition, $f^*[f_*[\g A]] \p\subseteq \g A$ is equivalent to $\a\langle$ for every $x \p\in f^*[f_*[\g A]] \a\langle$ $x \p\in \g A$ $\a\rangle$ $\a\rangle$. \par
    $|$\hs$|$\hs \psc{Since} \psc{We show} that $f^*[f_*[\g A]] \p\subseteq \g A$, \par
    $|$\hs$|$\hs \psc{and} $f^*[f_*[\g A]] \p\subseteq \g A$ is equivalent to $\a\langle$ for every $x \p\in f^*[f_*[\g A]] \a\langle$ $x \p\in \g A$ $\a\rangle$ $\a\rangle$, \par
    $|$\hs$|$\hs \psc{then}, by replacement, \psc{We show} that for every $x \p\in f^*[f_*[\g A]] \a\langle$ $x \p\in \g A$ $\a\rangle$. \par
    $|$\hs$|$\hs \psc{We show} that for every $x \p\in f^*[f_*[\g A]] \a\langle$ $x \p\in \g A$ $\a\rangle$. \par

      $|$\hs$|$\hs$|$\hs \par
      $|$\hs$|$\hs$|$\hs \psc{Let} $x \p\in \g A$ be an element of $f^*[f_*[\g A]]$. \par
      $|$\hs$|$\hs$|$\hs \psc{We show} that $x \p\in \g A$. \par

        $|$\hs$|$\hs$|$\hs$|$\hs \par
        $|$\hs$|$\hs$|$\hs$|$\hs By the preimage definition, $f^*[f_*[\g A]]$ is the set $\a\{x \p\in X \pipe \a\exists b_x \p\in f_*[\g A] \a\langle$ $f: x   \mapsto b_x$ $\a\rangle \a\}$. \par
        $|$\hs$|$\hs$|$\hs$|$\hs By the image definition,        $f_*[\g A]$  is the set $\a\{y \p\in Y \pipe \a\exists a_y \p\in     \g A  \a\langle$ $f: a_y \mapsto y$   $\a\rangle \a\}$. \par
        $|$\hs$|$\hs$|$\hs$|$\hs \psc{Since} $f: X \to Y$ is an injection, \par
        $|$\hs$|$\hs$|$\hs$|$\hs \psc{then}, by the injection definition, for every $x_0,x_1 \p\in X$ and $y \p\in Y \a\langle$ \asc{if} $f:x_0 \mapsto y$ \asc{and} $f: x_1 \mapsto y$, \asc{then} $x_0 \p= x_1$ $\a\rangle$. \par
        $|$\hs$|$\hs$|$\hs$|$\hs \psc{Since} $x$   is in $f^*[f_*[\g A]]$, \psc{then}, by the preimage definition, there exists $b_x \p\in f_*[\g A]$ so that $f: x   \mapsto b_x$. \par
        $|$\hs$|$\hs$|$\hs$|$\hs \psc{Since} $b_x$ is in     $f_*[\g A]$,  \psc{then}, by the image definition,    there exists $a_y \p\in     \g A$  so that $f: a_y \mapsto b_x$. \par
        $|$\hs$|$\hs$|$\hs$|$\hs \psc{Since} $x,a_y \p\in X$, \psc{and} $b_x \p\in Y$ \par
        $|$\hs$|$\hs$|$\hs$|$\hs \psc{and}   $f: x   \mapsto b_x$, \par
        $|$\hs$|$\hs$|$\hs$|$\hs \psc{and}   $f: a_y \mapsto b_x$, \par
        $|$\hs$|$\hs$|$\hs$|$\hs \psc{and}   for every $x_0,x_1 \p\in X$ and $y \p\in Y \a\langle$ \asc{if} $f:x_0 \mapsto y$ \asc{and} $f: x_1 \mapsto y$, \asc{then} $x_0 \p= x_1$ $\a\rangle$, \par
        $|$\hs$|$\hs$|$\hs$|$\hs \psc{then}, by setting $x_0 \p\leftarrow x$ and $x_1 \p\leftarrow a_y$ and $y \p\leftarrow b_x$, $x$ is equal to $a_y$. \par
        $|$\hs$|$\hs$|$\hs$|$\hs \psc{Since} $x \p= a_y$, \psc{and} $a_y \p\in \g A$ \psc{then}, by replacement, $x \p\in \g A$. \par
        $|$\hs$|$\hs$|$\hs$|$\hs \par

      $|$\hs$|$\hs$|$\hs \psc{This shows} that $x \p\in \g A$. \par
      $|$\hs$|$\hs$|$\hs \par

    $|$\hs$|$\hs \psc{This shows} that for every $x \p\in f^*[f_*[\g A]] \a\langle$ $x \p\in \g A$ $\a\rangle$. \par
    $|$\hs$|$\hs \par

  $|$\hs \psc{This shows} that $f^*[f_*[\g A]] \p\subseteq \g A$. \par
  $|$\hs \par

\psc{This shows} that for every $\g A \p\subseteq X \a\langle$ $f^*[f_*[\g A]] \p\subseteq \g A$ $\a\rangle$.
\fi

% ------------------------------------------------------------------------------------------------------------------------------
\ifnum 1=1
\vs
\asc{proof} of {\bf 2)}, if. \par
\psc{Let} $X,Y$ be sets. \par
\psc{Let} $f: X \to Y$ be a function. \par
\psc{Let} $f: X \to Y$ satisfy $\a\langle$ for every $\g A \p\subseteq X \a\langle$ $f^*[f_*[\g A]] \p\subseteq \g A$ $\a\rangle$ $\a\rangle$. \par
\psc{We show} that $f: X \to Y$ is injective. \par

  $|$\hs \par
  $|$\hs \psc{Since} \psc{We show} that $f$ is injective, \par
  $|$\hs \psc{then}, by the injective definition, \psc{We show} that for every $x_0,x_1 \p\in X$ and $y \p\in Y \a\langle$ \asc{if} $f: x_0 \mapsto y$ \asc{and} $f: x_1 \mapsto y$, \asc{then} $x_0 \p= x_1$ $\a\rangle$. \par
  $|$\hs \psc{We show} that for every $x_0,x_1 \p\in X$ and $y \p\in Y \a\langle$ \asc{if} $f: x_0 \mapsto y$ \asc{and} $f: x_1 \mapsto y$, \asc{then} $x_0 \p= x_1$ $\a\rangle$. \par

    $|$\hs$|$\hs \par
    $|$\hs$|$\hs \psc{Let} $x_0,x_1 \p\in X$. \par
    $|$\hs$|$\hs \psc{Let} $y \p\in Y$. \par
    $|$\hs$|$\hs \psc{We show} that \asc{if} $f: x_0 \mapsto y$ \asc{and} $f: x_1 \mapsto y$, \asc{then} $x_0 \p= x_1$. \par

      $|$\hs$|$\hs$|$\hs \par
      $|$\hs$|$\hs$|$\hs \psc{Let} $f: x_0 \mapsto y$ \asc{and} $f: x_1 \mapsto y$. \par
      $|$\hs$|$\hs$|$\hs \psc{We show} that $x_0 \p= x_1$. \par
        
        $|$\hs$|$\hs$|$\hs$|$\hs \par
        $|$\hs$|$\hs$|$\hs$|$\hs TODO
        % $|$\hs$|$\hs$|$\hs$|$\hs \psc{Since} $\g A \p\subseteq X \a\langle$ $f^*[f_*[\g A]] \p\subseteq \g A$ $\a\rangle$, \psc{and} $X \p\subseteq X$, \psc{then}, by setting $A \p\leftarrow X$,  $f^*[f_*[X]] \p\subseteq X$. \par
        % $|$\hs$|$\hs$|$\hs$|$\hs By the image definition,        $f_*[\g X]$  is the set $\a\{y \p\in Y \pipe \a\exists x_y \p\in     X \a\langle$ $f:x_y \mapsto y$   $\a\rangle \a\}$. \par
        % $|$\hs$|$\hs$|$\hs$|$\hs By the preimage definition, $f^*[f_*[\g X]]$ is the set $\a\{x \p\in X \pipe \a\exists y_x \p\in f_*[X] \a\langle$ $f:x   \mapsto y_x$ $\a\rangle \a\}$. \par
        % $|$\hs$|$\hs$|$\hs$|$\hs By the fundamental theorem of preimages, $f^*[f_*[X]] \p\supseteq X$. \par

        % $|$\hs$|$\hs$|$\hs$|$\hs \psc{Since} $f^*[f_*[X]] \p\supseteq X$, \psc{and} $x_0 \p\in X$, \psc{then} $x_0 \p\in f^*[f_*[X]]$. \par
        % $|$\hs$|$\hs$|$\hs$|$\hs \psc{Since} $f^*[f_*[X]] \p\supseteq X$, \psc{and} $x_1 \p\in X$, \psc{then} $x_1 \p\in f^*[f_*[X]]$. \par
        % $|$\hs$|$\hs$|$\hs$|$\hs \psc{Since} $x_0 \p\in f^*[f_*[X]]$, \psc{then}, by the preimage definition, there exists $y_0    \p\in f_*[X] \a\langle$ $f: x_0    \mapsto y_0$ $\a\rangle$. \par
        % $|$\hs$|$\hs$|$\hs$|$\hs \psc{Since} $x_1 \p\in f^*[f_*[X]]$, \psc{then}, by the preimage definition, there exists $y_1    \p\in f_*[X] \a\langle$ $f: x_1    \mapsto y_1$ $\a\rangle$. \par

        % $|$\hs$|$\hs$|$\hs$|$\hs \psc{Since} $f: x_0 \mapsto y_0$, \par
        % $|$\hs$|$\hs$|$\hs$|$\hs \psc{and}   $f: x_1 \mapsto y_1$, \par
        % $|$\hs$|$\hs$|$\hs$|$\hs \psc{and}   $f: x_0 \mapsto y$, \par
        % $|$\hs$|$\hs$|$\hs$|$\hs \psc{and}   $f: x_1 \mapsto y$, \par
        % $|$\hs$|$\hs$|$\hs$|$\hs \psc{then}, by the uniqueness of images, $y_0 \p= y \p= y1$. \par

        % $|$\hs$|$\hs$|$\hs$|$\hs \psc{Since} $y \p\in f_*[X]$,  \psc{then}, by the image definition, there exists $x_{00} \p\in X \a\langle$ $f: x_{00} \mapsto y$ $\a\rangle$. \par
        % $|$\hs$|$\hs$|$\hs$|$\hs \psc{Since} $y \p\in f_*[X]$,  \psc{then}, by the image definition, there exists $x_{11} \p\in X \a\langle$ $f: x_{11} \mapsto y$ $\a\rangle$. \par
        % $|$\hs$|$\hs$|$\hs$|$\hs \psc{Since} $x_{00} \p\in X$ and . \par
        % $|$\hs$|$\hs$|$\hs$|$\hs \psc{Since} $x_{11} \p\in X$ and . \par
        $|$\hs$|$\hs$|$\hs$|$\hs \par

      $|$\hs$|$\hs$|$\hs \psc{This shows} that $x_0 \p= x_1$. \par
      $|$\hs$|$\hs$|$\hs \par

    $|$\hs$|$\hs \psc{This shows} that \asc{if} $f: x_0 \mapsto y$ \asc{and} $f: x_1 \mapsto y$, \asc{then} $x_0 \p= x_1$. \par
    $|$\hs$|$\hs \par

  $|$\hs \psc{This shows} that for every $x_0,x_1 \p\in X$ and $y \p\in Y \a\langle$ \asc{if} $f: x_0 \mapsto y$ \asc{and} $f: x_1 \mapsto y$, \asc{then} $x_0 \p= x_1$ $\a\rangle$. \par

\vs
\psc{This shows} that $f: X \to Y$ is injective. \par
\fi

% ------------------------------------------------------------------------------------------------------------------------------
\ifnum 1=1
\vs
\asc{proof} of {\bf 3)}, only if. \par
TODO
\fi

% ------------------------------------------------------------------------------------------------------------------------------
\ifnum 1=1
\vs
\asc{proof} of {\bf 3)}, if. \par
TODO
\fi


% ------------------------------------------------------------------------------------------------------------------------------
% ------------------------------------------------------------------------------------------------------------------------------
\vs\hrule\vskip1pt
\theorem The fundamental meta-theorem of equations. \par
Let $A$ be a ``math expression''. \par
Let $B$ be a ``math expression''. \par
Let $f$ be a function from ``math expressions'' to ``math expressions'' (ie. the image under $f$ of each ``math expression'' is unique). \par
  \hs{\bf 0)} \asc{If} $A$ {\bf equals} $B$, \asc{then} $f[A]$ {\bf equals} $f[B]$. \par

\ifnum 1=1
\asc{proof}. I don't know.
\fi




% ------------------------------------------------------------------------------------------------------------------------------
% ------------------------------------------------------------------------------------------------------------------------------
% ------------------------------------------------------------------------------------------------------------------------------
% ------------------------------------------------------------------------------------------------------------------------------
\chapter{Convergence, pillar of analysis}

% ------------------------------------------------------------------------------------------------------------------------------
\vs
{\bf Limits} are the workhorse of {\bf analysis}. In analysis, ``everything is a limit``. Or something.

\vs
{\bf Derivatives} are limits. {\bf Integrals} are limits. {\bf Continuity} is defined using limits. Even {\bf equality} can be defined using limits (kinda).

\vs
But we'll prefer a different language: that of {\bf convergence}. Limits and convergence are the same idea. \par
You can say that analysis is built on {\bf limits}. \par
You can say that analysis is built on {\bf convergence}. \par

% ------------------------------------------------------------------------------------------------------------------------------
\vs\hrule\vskip1pt
\definition {\bf Limits} and {\bf convergence} of sequences. \par  % precision/accuracy --> wiggle/threshold/tolerance
Let $f: \N \to \R$ be a sequence. \par
Let $L \p\in \R$   be a real number in the codomain of $f$. \par
  \hs{\bf 0)} $f$ has \abf{limit} $L$ (at {\it infinity}), denoted $f \to L$, \asc{iff} \par
      \hs\hs\hs for every precision $\epsilon \p\in \R^+$ \par
      \hs\hs\hs\hs there exists a threshold $N_\epsilon \p\in \N$ so that \par
      \hs\hs\hs\hs\hs for every $x \p\in \N$ in the domain $\a\langle$ \par
      \hs\hs\hs\hs\hs\hs \asc{if} $x$ is in the $\infty$-ball $\a(N_\epsilon \a{..} \infty\a)_\N$, \asc{then} $f[x]$ is in the $\epsilon$-ball $\a(L\p-\epsilon \a{..} L\p+\epsilon\a)_\R$ \par
      \hs\hs\hs\hs\hs $\a\rangle$. \par
  \hs{\bf 1)} $f$ \abf{converges} to $L$ (at {\it infinity}), denoted $f \to L$, \asc{iff} \par
      \hs\hs\hs for every precision $\epsilon \p\in \R^+$ \par
      \hs\hs\hs\hs there exists a threshold $N_\epsilon \p\in \N$ so that \par
      \hs\hs\hs\hs\hs for every $x \p\in \N$ in the domain $\a\langle$ \par
      \hs\hs\hs\hs\hs\hs \asc{if} $x$ is in the $\infty$-ball $\a(N_\epsilon \a{..} \infty\a)_\N$, \asc{then} $f[x]$ is in the $\epsilon$-ball $\a(L\p-\epsilon \a{..} L\p+\epsilon\a)_\R$ \par
      \hs\hs\hs\hs\hs $\a\rangle$. \par
  \hs{\bf 2)} $f$ has \abf{limit} $L$ (at {\it infinity}) \asc{iff} $f$ \abf{converges} to $L$ (at {\it infinity}). \par

\vs
\definition {\bf Limits} and {\bf convergence} of functions. \par  % precision/accuracy --> wiggle/threshold/tolerance
Let $f: A\p\subseteq\R \to \R$ be function. \par
Let $a \p\in A$  be a real number in the domain   of $f$. \par
Let $L \p\in \R$ be a real number in the codomain of $f$. \par
  \hs{\bf 0)} $f$ has \abf{limit} $L$ at $a$, denoted $f \to L \p@ a$, \asc{iff} \par
      \hs\hs\hs for every precision $\epsilon \p\in \R^+$ \par
      \hs\hs\hs\hs there exists a threshold $\delta_\epsilon \p\in \R^+$ so that \par
      \hs\hs\hs\hs\hs for every $x \p\in A$ in the domain $\a\langle$ \par
      \hs\hs\hs\hs\hs\hs \asc{if} $x$ is in the $\delta$-ball $\a(a\p-\delta_\epsilon \a{..} a\p+\delta_\epsilon\a)_\R$, \asc{then} $f[x]$ is in the $\epsilon$-ball $\a(L\p-\epsilon \a{..} L\p+\epsilon\a)_\R$ \par
      \hs\hs\hs\hs\hs $\a\rangle$. \par
  \hs{\bf 1)} $f$ \abf{converges} to $L$ at $a$, denoted $f \to L \p@ a$, \asc{iff} \par
      \hs\hs\hs for every precision $\epsilon \p\in \R^+$ \par
      \hs\hs\hs\hs there exists a threshold $\delta_\epsilon \p\in \R^+$ so that \par
      \hs\hs\hs\hs\hs for every $x \p\in A$ in the domain $\a\langle$ \par
      \hs\hs\hs\hs\hs\hs \asc{if} $x$ is in the $\delta$-ball $\a(a\p-\delta_\epsilon \a{..} a\p+\delta_\epsilon\a)_\R$, \asc{then} $f[x]$ is in the $\epsilon$-ball $\a(L\p-\epsilon \a{..} L\p+\epsilon\a)_\R$ \par
      \hs\hs\hs\hs\hs $\a\rangle$. \par
  \hs{\bf 2)} $f$ has \abf{limit} $L$ at $a$ \asc{iff} $f$ \abf{converges} to $L$ at $a$. \par



% ------------------------------------------------------------------------------------------------------------------------------
% ------------------------------------------------------------------------------------------------------------------------------
% ------------------------------------------------------------------------------------------------------------------------------
\section{Open sets, a language for convergence}

% ------------------------------------------------------------------------------------------------------------------------------
\vs
\theorem {\bf Convergence} via open sets. \par
Let $f: A\p\subseteq\R \to \R$ be function. \par
Let $a \p\in A$  be a real number in the domain   of $f$. \par
Let $L \p\in \R$ be a real number in the codomain of $f$. \par
  \hs{\bf 0)} $f$ \abf{converges} to $L$ at $a$ \asc{iff} for every $\epsilon \p\in \R^+$, there exists $\delta_\epsilon \p\in \R^+$ so that, for every $x \p\in A$ in the domain $\a\langle$ \par
      \hs\hs\hs \asc{if} $x \p\in \a(a\p-\delta_\epsilon \a{..} a\p+\delta_\epsilon\a)_\R$, \asc{then} $f[x] \p\in \a(L\p-\epsilon \a{..} L\p+\epsilon\a)_\R$ \par
    \hs\hs $\a\rangle$. \par
  \hs{\bf 1)} $f$ \abf{converges} to $L$ at $a$ \asc{iff} for every $\epsilon \p\in \R^+$, there exists $\delta_\epsilon \p\in \R^+$ so that, for every $x \p\in A$ in the domain $\a\langle$ \par
      \hs\hs\hs \asc{if} $x \p\in \a(a\p-\delta_\epsilon \a{..} a\p+\delta_\epsilon\a)_\R$, \asc{then} $x \p\in f^*[\a(L\p-\epsilon \a{..} L\p+\epsilon\a)_\R]$ \par
    \hs\hs $\a\rangle$. \par
  \hs{\bf 2)} $f$ \abf{converges} to $L$ at $a$ \asc{iff} for every $\epsilon \p\in \R^+$, there exists $\delta_\epsilon \p\in \R^+ \a\langle$ \par
      \hs\hs\hs $\a(a\p-\delta_\epsilon \a{..} a\p+\delta_\epsilon\a)_\R$ $\p\subseteq$ $f^*[\a(L\p-\epsilon \a{..} L\p+\epsilon\a)_\R]$ \par
    \hs\hs $\a\rangle$. \par
  % \hs{\bf 3)} $f$ \abf{converges} to $L$ at $a$ \asc{iff} for every $\epsilon \p\in \R^+$, there exists $\delta_\epsilon \p\in \R^+ \a\langle$ \par
  %     \hs\hs\hs $f_*[\a(a\p-\delta_\epsilon \a{..} a\p+\delta_\epsilon\a)_\R]$ $\p\subseteq$ $f_*[f^*[\a(L\p-\epsilon \a{..} L\p+\epsilon\a)_\R]]$ \par
  %   \hs\hs $\a\rangle$. \par
  % \hs{\bf 4)} $f$ \abf{converges} to $L$ at $a$ \asc{iff} for every $\epsilon \p\in \R^+$, there exists $\delta_\epsilon \p\in \R^+ \a\langle$ \par
  %     \hs\hs\hs $f_*[\a(a\p-\delta_\epsilon \a{..} a\p+\delta_\epsilon\a)_\R]$ $\p\subseteq$ $f_*[f^*[\a(L\p-\epsilon \a{..} L\p+\epsilon\a)_\R]]$ $\p\subseteq$ $\a(L\p-\epsilon \a{..} L\p+\epsilon\a)_\R$ \par
  %   \hs\hs $\a\rangle$. \par
  \hs{\bf 3)} $f$ \abf{converges} to $L$ at $a$ \asc{iff} for every $\epsilon \p\in \R^+$, there exists $\delta_\epsilon \p\in \R^+ \a\langle$ \par
      \hs\hs\hs $f_*[\a(a\p-\delta_\epsilon \a{..} a\p+\delta_\epsilon\a)_\R]$ $\p\subseteq$ $\a(L\p-\epsilon \a{..} L\p+\epsilon\a)_\R$ \par
    \hs\hs $\a\rangle$. \par
  % \hs{\bf 6)} $f$ \abf{converges} to $L$ at $a$ \asc{iff} for every $\epsilon \p\in \R^+$, there exists $\delta_\epsilon \p\in \R^+ \a\langle$ \par
  %     \hs\hs\hs $f^*[f_*[\a(a\p-\delta_\epsilon \a{..} a\p+\delta_\epsilon\a)_\R]]$ $\p\subseteq$ $f^*[\a(L\p-\epsilon \a{..} L\p+\epsilon\a)_\R]$ \par
  %   \hs\hs $\a\rangle$. \par
  % \hs{\bf 7)} $f$ \abf{converges} to $L$ at $a$ \asc{iff} for every $\epsilon \p\in \R^+$, there exists $\delta_\epsilon \p\in \R^+ \a\langle$ \par
  %     \hs\hs\hs $\a(a\p-\delta_\epsilon \a{..} a\p+\delta_\epsilon\a)_\R$ $\p\subseteq$ $f^*[f_*[\a(a\p-\delta_\epsilon \a{..} a\p+\delta_\epsilon\a)_\R]]$ $\p\subseteq$ $f^*[\a(L\p-\epsilon \a{..} L\p+\epsilon\a)_\R]$ \par
  %   \hs\hs $\a\rangle$. \par
  % \hs{\bf 8)} $f$ \abf{converges} to $L$ at $a$ \asc{iff} for every $\epsilon \p\in \R^+$, there exists $\delta_\epsilon \p\in \R^+ \a\langle$ \par
  %     \hs\hs\hs $\a(a\p-\delta_\epsilon \a{..} a\p+\delta_\epsilon\a)_\R$ $\p\subseteq$ $f^*[\a(L\p-\epsilon \a{..} L\p+\epsilon\a)_\R]$ \par
  %   \hs\hs $\a\rangle$. \par

  \hs{\bf 4)} $f$ \abf{converges} to $L$ at $a$ \asc{iff} for every open ball $B[L,\epsilon]$ at $L$, there exists an open ball $B[a,\delta_\epsilon]$ at $a \a\langle$ \par
      \hs\hs\hs $B[a,\delta_\epsilon]$ $\p\subseteq$ $f^*[B[L,\epsilon]]$ \par
    \hs\hs $\a\rangle$. \par
  \hs{\bf 5)} $f$ \abf{converges} to $L$ at $a$ \asc{iff} for every open ball $B[L,\epsilon]$ at $L$, there exists an open ball $B[a,\delta_\epsilon]$ at $a \a\langle$ \par
      \hs\hs\hs $f_*[B[a,\delta_\epsilon]]$ $\p\subseteq$ $B[L,\epsilon]$ \par
    \hs\hs $\a\rangle$. \par

  \hs{\bf 6)} $f$ \abf{converges} to $L$ at $a$ \asc{iff} for every open ball $B[L,\epsilon]$ at $L \a\langle$ \par
      \hs\hs\hs $f^*[B[L,\epsilon]]$ is open \par
    \hs\hs $\a\rangle$. \par

% ------------------------------------------------------------------------------------------------------------------------------
\vs
\asc{proof} of {\bf 0)}. \par
This is just the convergence definition, for reference $\p{=)}$

% ------------------------------------------------------------------------------------------------------------------------------
\vs
\asc{proof} of {\bf 1)}, only if. \par
Let $f: A\p\subseteq\R \to \R$ be function. \par
Let $a \p\in A$  be a real number in the domain   of $f$. \par
Let $L \p\in \R$ be a real number in the codomain of $f$. \par
Let $f$ \abf{converge} to $L$ at $a$. \par
\psc{We show} that for every $\epsilon \p\in \R^+$, there exists $\delta_\epsilon \p\in \R^+$ so that, for every $x \p\in A \a\langle$ \asc{if} $x \p\in \a(a\p-\delta_\epsilon \a{..} a\p+\delta_\epsilon\a)_\R$, \asc{then} $x \p\in f^*[\a(L\p-\epsilon \a{..} L\p+\epsilon\a)_\R]$ $\a\rangle$. \par

  \vs
  \hs By the fundamental lemma of functions, for every subset $B \p\subseteq {\bf Cod}[f]$, for every $x \p\in {\bf Dom}[f] \a\langle$ $f[x] \p\in B$ \asc{iff} $x \p\in f^*[B]$ $\a\rangle$. \par
  \hs \psc{Since} $\a(L\p-\epsilon \a{..} L\p+\epsilon\a)_\R$ is a subset of ${\bf Cod}[f]$, \psc{and} $x$ is in ${\bf Dom}[f]$, \par
  \hs \psc{then}, by the fundamental lemma of functions and setting $B \p\leftarrow \a(L\p-\epsilon \a{..} L\p+\epsilon\a)_\R$, we get that $f[x] \p\in \a(L\p-\epsilon \a{..} L\p+\epsilon\a)_\R$ \asc{iff} $x \p\in f^*[\a(L\p-\epsilon \a{..} L\p+\epsilon\a)_\R]$. \par

  \vs
  \hs \psc{Since} $f$ \abf{converges} to $L$ at $a$, \par
  \hs \psc{then}, by the convergence definition, \par
  \hs for every $\epsilon \p\in \R^+$, there exists $\delta_\epsilon \p\in \R^+$ so that, for every $x \p\in A \a\langle$ \asc{if} $x \p\in \a(a\p-\delta_\epsilon \a{..} a\p+\delta_\epsilon\a)_\N$, \asc{then} $f[x] \p\in \a(L\p-\epsilon \a{..} L\p+\epsilon\a)_\R$ $\a\rangle$. \par

  \vs
  \hs \psc{Since} $f[x] \p\in \a(L\p-\epsilon \a{..} L\p+\epsilon\a)_\R$ \asc{iff} $x \p\in f^*[\a(L\p-\epsilon \a{..} L\p+\epsilon\a)_\R]$, \par
  \hs \psc{and}   for every $\epsilon \p\in \R^+$, there exists $\delta_\epsilon \p\in \R^+$ so that, for every $x \p\in A \a\langle$ \asc{if} $x \p\in \a(a\p-\delta_\epsilon \a{..} a\p+\delta_\epsilon\a)_\N$, \asc{then} $f[x] \p\in \a(L\p-\epsilon \a{..} L\p+\epsilon\a)_\R$ $\a\rangle$, \par
  \hs \psc{then}, by replacing $f[x] \p\in \a(L\p-\epsilon \a{..} L\p+\epsilon\a)_\R$ with $x \p\in f^*[\a(L\p-\epsilon \a{..} L\p+\epsilon\a)_\R]$, \par
  \hs for every $\epsilon \p\in \R^+$, there exists $\delta_\epsilon \p\in \R^+$ so that, for every $x \p\in A \a\langle$ \asc{if} $x \p\in \a(a\p-\delta_\epsilon \a{..} a\p+\delta_\epsilon\a)_\R$, \asc{then} $x \p\in f^*[\a(L\p-\epsilon \a{..} L\p+\epsilon\a)_\R]$ $\a\rangle$. \par

\vs
\psc{This shows} that for every $\epsilon \p\in \R^+$, there exists $\delta_\epsilon \p\in \R^+$ so that, for every $x \p\in A \a\langle$ \asc{if} $x \p\in \a(a\p-\delta_\epsilon \a{..} a\p+\delta_\epsilon\a)_\R$, \asc{then} $x \p\in f^*[\a(L\p-\epsilon \a{..} L\p+\epsilon\a)_\R]$ $\a\rangle$. \par

% ------------------------------------------------------------------------------------------------------------------------------
\vs
\asc{proof} of {\bf 1)}, if. \par
Let $f: A\p\subseteq\R \to \R$ be function. \par
Let $a \p\in A$  be a real number in the domain   of $f$. \par
Let $L \p\in \R$ be a real number in the codomain of $f$. \par
Let $\a\langle$ for every $\epsilon \p\in \R^+$, there exists $\delta_\epsilon \p\in \R^+$ so that, for every $x \p\in A \a\langle$ \asc{if} $x \p\in \a(a\p-\delta_\epsilon \a{..} a\p+\delta_\epsilon\a)_\R$, \asc{then} $x \p\in f^*[\a(L\p-\epsilon \a{..} L\p+\epsilon\a)_\R]$ $\a\rangle$ $\a\rangle$. \par
\psc{We show} that  $f$ \abf{converges} to $L$ at $a$. \par

\hs \par
\hs TODO

\vs
\psc{This shows} that  $f$ \abf{converges} to $L$ at $a$. \par

% ------------------------------------------------------------------------------------------------------------------------------
\vs
\asc{proof} of {\bf 1)}, direct. \par
Let $f: A\p\subseteq\R \to \R$ be function. \par
Let $a \p\in A$  be a real number in the domain   of $f$. \par
Let $L \p\in \R$ be a real number in the codomain of $f$. \par
\psc{We show} that $\a\langle$ $f \to L \p@ a$ $\a\rangle$ \asc{iff} $\a\langle$ $\a\forall \epsilon \p\in \R^+$ $\a\exists \delta_\epsilon \p\in \R^+$ $\a\forall x \p\in A \a\langle$ $x \p\in \a(a\p-\delta_\epsilon \a{..} a\p+\delta_\epsilon\a)_\R$ $\a\lthen$ $x \p\in f^*[\a(L\p-\epsilon \a{..} L\p+\epsilon\a)_\R]$ $\a\rangle$ $\a\rangle$. \par

\hs \par
\hs By the convergence definition, \par
\hs $\a\langle$ $f \to L \p@ a$ $\a\rangle$ is equivalent to $\a\langle$ $\a\forall \epsilon \p\in \R^+$ $\a\exists \delta_\epsilon \p\in \R^+$ $\a\forall x \p\in A \a\langle$ $x \p\in \a(a\p-\delta_\epsilon \a{..} a\p+\delta_\epsilon\a)_\R$ $\a\lthen$ $f[x] \p\in \a(L\p-\epsilon \a{..} L\p+\epsilon\a)_\R$ $\a\rangle$ $\a\rangle$. \par
\hs \psc{Since} $\a\langle$ $f \to L \p@ a$ $\a\rangle$ is equivalent to $\a\langle$ $\a\forall \epsilon \p\in \R^+$ $\a\exists \delta_\epsilon \p\in \R^+$ $\a\forall x \p\in A \a\langle$ $x \p\in \a(a\p-\delta_\epsilon \a{..} a\p+\delta_\epsilon\a)_\R$ $\a\lthen$ $f[x] \p\in \a(L\p-\epsilon \a{..} L\p+\epsilon\a)_\R$ $\a\rangle$ $\a\rangle$, \par
\hs \psc{and} \psc{We show} that $\a\langle$ $f \to L \p@ a$ $\a\rangle$ \asc{iff} $\a\langle$ $\a\forall \epsilon \p\in \R^+$ $\a\exists \delta_\epsilon \p\in \R^+$ $\a\forall x \p\in A \a\langle$ $x \p\in \a(a\p-\delta_\epsilon \a{..} a\p+\delta_\epsilon\a)_\R$ $\a\lthen$ $x \p\in f^*[\a(L\p-\epsilon \a{..} L\p+\epsilon\a)_\R]$ $\a\rangle$ $\a\rangle$, \par
\hs \psc{then}, by replacement, \psc{We show} that \par
\hs $\a\langle$ $\a\forall \epsilon \p\in \R^+$ $\a\exists \delta_\epsilon \p\in \R^+$ $\a\forall x \p\in A \a\langle$ $x \p\in \a(a\p-\delta_\epsilon \a{..} a\p+\delta_\epsilon\a)_\R$ $\a\lthen$ $f[x] \p\in \a(L\p-\epsilon \a{..} L\p+\epsilon\a)_\R$ $\a\rangle$ $\a\rangle$ \par
\hs \asc{iff} \par
\hs $\a\langle$ $\a\forall \epsilon \p\in \R^+$ $\a\exists \delta_\epsilon \p\in \R^+$ $\a\forall x \p\in A \a\langle$ $x \p\in \a(a\p-\delta_\epsilon \a{..} a\p+\delta_\epsilon\a)_\R$ $\a\lthen$ $x \p\in f^*[\a(L\p-\epsilon \a{..} L\p+\epsilon\a)_\R]$ $\a\rangle$ $\a\rangle$. \par
\hs \psc{We show} that \par
\hs $\a\langle$ $\a\forall \epsilon \p\in \R^+$ $\a\exists \delta_\epsilon \p\in \R^+$ $\a\forall x \p\in A \a\langle$ $x \p\in \a(a\p-\delta_\epsilon \a{..} a\p+\delta_\epsilon\a)_\R$ $\a\lthen$ $f[x] \p\in \a(L\p-\epsilon \a{..} L\p+\epsilon\a)_\R$ $\a\rangle$ $\a\rangle$ \par
\hs \asc{iff} \par
\hs $\a\langle$ $\a\forall \epsilon \p\in \R^+$ $\a\exists \delta_\epsilon \p\in \R^+$ $\a\forall x \p\in A \a\langle$ $x \p\in \a(a\p-\delta_\epsilon \a{..} a\p+\delta_\epsilon\a)_\R$ $\a\lthen$ $x \p\in f^*[\a(L\p-\epsilon \a{..} L\p+\epsilon\a)_\R]$ $\a\rangle$ $\a\rangle$. \par

\hs\hs \par
\hs\hs \psc{Since} \psc{We show} that \par
\hs\hs $\a\langle$ $\a\forall \epsilon \p\in \R^+$ $\a\exists \delta_\epsilon \p\in \R^+$ $\a\forall x \p\in A \a\langle$ $x \p\in \a(a\p-\delta_\epsilon \a{..} a\p+\delta_\epsilon\a)_\R$ $\a\lthen$ $f[x] \p\in \a(L\p-\epsilon \a{..} L\p+\epsilon\a)_\R$ $\a\rangle$ $\a\rangle$ \par
\hs\hs \asc{iff} \par
\hs\hs $\a\langle$ $\a\forall \epsilon \p\in \R^+$ $\a\exists \delta_\epsilon \p\in \R^+$ $\a\forall x \p\in A \a\langle$ $x \p\in \a(a\p-\delta_\epsilon \a{..} a\p+\delta_\epsilon\a)_\R$ $\a\lthen$ $x \p\in f^*[\a(L\p-\epsilon \a{..} L\p+\epsilon\a)_\R]$ $\a\rangle$ $\a\rangle$. \par
\hs\hs \psc{then}, by the fundamental lemma of first-order classical logic, we can peel the outer quantifier layers that are equal, so \par
\hs\hs \psc{We show} that \par
\hs\hs $x \p\in \a(a\p-\delta_\epsilon \a{..} a\p+\delta_\epsilon\a)_\R$ $\a\lthen$ $f[x] \p\in \a(L\p-\epsilon \a{..} L\p+\epsilon\a)_\R$ \par
\hs\hs \asc{iff} \par
\hs\hs $x \p\in \a(a\p-\delta_\epsilon \a{..} a\p+\delta_\epsilon\a)_\R$ $\a\lthen$ $x \p\in f^*[\a(L\p-\epsilon \a{..} L\p+\epsilon\a)_\R]$. \par
\hs\hs \psc{We show} that \par
\hs\hs $x \p\in \a(a\p-\delta_\epsilon \a{..} a\p+\delta_\epsilon\a)_\R$ $\a\lthen$ $f[x] \p\in \a(L\p-\epsilon \a{..} L\p+\epsilon\a)_\R$ \par
\hs\hs \asc{iff} \par
\hs\hs $x \p\in \a(a\p-\delta_\epsilon \a{..} a\p+\delta_\epsilon\a)_\R$ $\a\lthen$ $x \p\in f^*[\a(L\p-\epsilon \a{..} L\p+\epsilon\a)_\R]$. \par

\hs\hs\hs \par
\hs\hs\hs \psc{Let} $x \p\in \a(a\p-\delta_\epsilon \a{..} a\p+\delta_\epsilon\a)_\R$. \par
\hs\hs\hs \psc{We show} that \par
\hs\hs\hs $f[x] \p\in \a(L\p-\epsilon \a{..} L\p+\epsilon\a)_\R$ \par
\hs\hs\hs \asc{iff} \par
\hs\hs\hs $x \p\in f^*[\a(L\p-\epsilon \a{..} L\p+\epsilon\a)_\R]$. \par

\hs\hs\hs\hs \par
\hs\hs\hs\hs By the fundamental lemma of functions, setting $B$ to $\a(L\p-\epsilon \a{..} L\p+\epsilon\a)_\R$, we get that \par
\hs\hs\hs\hs $f[x] \p\in B$ \asc{iff} $x \p\in f^*[B]$, meaning \par
\hs\hs\hs\hs $f[x] \p\in \a(L\p-\epsilon \a{..} L\p+\epsilon\a)_\R$ \asc{iff} $x \p\in f^*[\a(L\p-\epsilon \a{..} L\p+\epsilon\a)_\R]$. \par

\hs\hs\hs \par
\hs\hs\hs \psc{This shows} that \par
\hs\hs\hs $f[x] \p\in \a(L\p-\epsilon \a{..} L\p+\epsilon\a)_\R$ \par
\hs\hs\hs \asc{iff} \par
\hs\hs\hs $x \p\in f^*[\a(L\p-\epsilon \a{..} L\p+\epsilon\a)_\R]$. \par

\hs\hs \par
\hs\hs \psc{This shows} that \par
\hs\hs $x \p\in \a(a\p-\delta_\epsilon \a{..} a\p+\delta_\epsilon\a)_\R$ $\a\lthen$ $f[x] \p\in \a(L\p-\epsilon \a{..} L\p+\epsilon\a)_\R$ \par
\hs\hs \asc{iff} \par
\hs\hs $x \p\in \a(a\p-\delta_\epsilon \a{..} a\p+\delta_\epsilon\a)_\R$ $\a\lthen$ $x \p\in f^*[\a(L\p-\epsilon \a{..} L\p+\epsilon\a)_\R]$. \par

\hs \par
\hs \psc{This shows} that \par
\hs $\a\langle$ $\a\forall \epsilon \p\in \R^+$ $\a\exists \delta_\epsilon \p\in \R^+$ $\a\forall x \p\in A \a\langle$ $x \p\in \a(a\p-\delta_\epsilon \a{..} a\p+\delta_\epsilon\a)_\R$ $\a\lthen$ $f[x] \p\in \a(L\p-\epsilon \a{..} L\p+\epsilon\a)_\R$ $\a\rangle$ $\a\rangle$ \par
\hs \asc{iff} \par
\hs $\a\langle$ $\a\forall \epsilon \p\in \R^+$ $\a\exists \delta_\epsilon \p\in \R^+$ $\a\forall x \p\in A \a\langle$ $x \p\in \a(a\p-\delta_\epsilon \a{..} a\p+\delta_\epsilon\a)_\R$ $\a\lthen$ $x \p\in f^*[\a(L\p-\epsilon \a{..} L\p+\epsilon\a)_\R]$ $\a\rangle$ $\a\rangle$. \par

\vs
\psc{This shows} that $\a\langle$ $f \to L \p@ a$ $\a\rangle$ \asc{iff} $\a\langle$ $\a\forall \epsilon \p\in \R^+$ $\a\exists \delta_\epsilon \p\in \R^+$ $\a\forall x \p\in A \a\langle$ $x \p\in \a(a\p-\delta_\epsilon \a{..} a\p+\delta_\epsilon\a)_\R$ $\a\lthen$ $x \p\in f^*[\a(L\p-\epsilon \a{..} L\p+\epsilon\a)_\R]$ $\a\rangle$ $\a\rangle$. \par

% ------------------------------------------------------------------------------------------------------------------------------
\vs
\asc{proof} of {\bf 2)}. \par
Let $f: A\p\subseteq\R \to \R$ be function. \par
Let $a \p\in A$  be a real number in the domain   of $f$. \par
Let $L \p\in \R$ be a real number in the codomain of $f$. \par
TODO

% ------------------------------------------------------------------------------------------------------------------------------
\vs
\asc{proof} of {\bf 3)}. \par
Let $f: A\p\subseteq\R \to \R$ be function. \par
Let $a \p\in A$  be a real number in the domain   of $f$. \par
Let $L \p\in \R$ be a real number in the codomain of $f$. \par
TODO

% ------------------------------------------------------------------------------------------------------------------------------
\vs
\asc{proof} of {\bf 4)}. \par
Let $f: A\p\subseteq\R \to \R$ be function. \par
Let $a \p\in A$  be a real number in the domain   of $f$. \par
Let $L \p\in \R$ be a real number in the codomain of $f$. \par
TODO

% ------------------------------------------------------------------------------------------------------------------------------
\vs
\asc{proof} of {\bf 5)}. \par
Let $f: A\p\subseteq\R \to \R$ be function. \par
Let $a \p\in A$  be a real number in the domain   of $f$. \par
Let $L \p\in \R$ be a real number in the codomain of $f$. \par
TODO

% ------------------------------------------------------------------------------------------------------------------------------
\vs
\asc{proof} of {\bf 6)}. \par
Let $f: A\p\subseteq\R \to \R$ be function. \par
Let $a \p\in A$  be a real number in the domain   of $f$. \par
Let $L \p\in \R$ be a real number in the codomain of $f$. \par
TODO



% ------------------------------------------------------------------------------------------------------------------------------
% ------------------------------------------------------------------------------------------------------------------------------
% ------------------------------------------------------------------------------------------------------------------------------
\section{The fundamental theorem of $\epsilon$-equality}

% ------------------------------------------------------------------------------------------------------------------------------
\definition Let $a,b \p\in \R$ be real numbers. \par
\halign{#\hfil & #\hfil \cr
  \hs{\bf 0)} $a$ is {\bf under}    & $b$ \asc{iff} $a \p< b$. \cr
  \hs{\bf 1)} $a$ is {\bf over}     & $b$ \asc{iff} $a \p> b$. \cr
  \hs{\bf 2)} $a$ is {\bf at most}  & $b$ \asc{iff} $a \p\leq b$. \cr
  \hs{\bf 3)} $a$ is {\bf at least} & $b$ \asc{iff} $a \p\geq b$. \cr
}

% ------------------------------------------------------------------------------------------------------------------------------
\vs
\lemma Let $a,b \p\in \R$ be real numbers. \par
  \hs{\bf 0)} \asc{if} for every positive $\epsilon\p\in \R^+$ it's true that $\a|a \p- b\a| \p< \epsilon$, \asc{then} $\a|a \p- b\a| \p\leq 0$. \par
  \hs{\bf 1)} \asc{if} for every positive $\epsilon\p\in \R^+$ it's true that $\a|a \p- b\a| \p< \epsilon$, \asc{then} $\a|a \p- b\a| \p\notin \R^+$. \par
  \hs{\bf 2)} \asc{if} for every positive $\epsilon\p\in \R^+$ it's true that $\a|a \p- b\a| - \epsilon \p\in \R^-$, \asc{then} $\a|a \p- b\a| \p\notin \R^+$. \par
    % \hs\hs By the law of the excluded middle, $\a\forall \epsilon \p\in \R^+ \a\langle \a|a \p- b\a| \a\rangle$ is {\bf equivalent} to its double negation $\neg\neg\a\forall \epsilon \p\in \R^+ \a\langle \a|a \p- b\a| \p< \epsilon \a\rangle$. \par
    % \hs\hs By the rules of classical logic, $\neg\neg\a\forall \epsilon \p\in \R^+ \a\langle \a|a \p- b\a| \p< \epsilon \a\rangle$ is equivalent to $\neg\a\exists \epsilon \p\in \R^+ \neg\a\langle \a|a \p- b\a| \p< \epsilon \a\rangle$. \par
    % \hs\hs By the $\p<$ relation definition, the negation of $\a|a \p- b\a| \p< \epsilon$ is $\a|a \p- b\a| \p\geq \epsilon$. \par
    % \hs\hs \psc{Since} the negation of $\a|a \p- b\a| \p< \epsilon$ is $\a|a \p- b\a| \p\geq \epsilon$, \psc{then} $\neg\a\exists \epsilon \p\in \R^+ \neg\a\langle \a|a \p- b\a| \p< \epsilon \a\rangle$ is equivalent to $\neg\a\exists \epsilon \p\in \R^+ \a\langle \a|a \p- b\a| \p\geq \epsilon \a\rangle$. \par

% ------------------------------------------------------------------------------------------------------------------------------
\vs
\theorem The fundamental theorem of $\epsilon$-equality, aka the fundamental theorem of analytic equality. \par
Let $a,b \p\in \R$ be real numbers. \par
  \hs{\bf 0)} $a$ {\bf equals} $b$ \asc{iff} for every positive $\epsilon\p\in \R^+$ it's true that $\a|a \p- b\a| \p< \epsilon$. \par
In symbols, \par

  \hs for every $a,b \p\in \R \a\langle$ \par
    \hs\hs $a\p=b$ \asc{iff} for every $\epsilon \p\in \R^+ \a\langle$ \par
      \hs\hs\hs $\a|a \p- b\a| \p< \epsilon$ \par
    \hs\hs $\a\rangle$ \par
  \hs $\a\rangle$. \par

\asc{proof}. \par
\a{H0)} \psc{Let} $a,b \p\in \R$ be real numbers. \par
\psc{We show} that $a$ {\bf equals} $b$ \asc{iff} for every positive $\epsilon\p\in \R^+$ it's true that $\a|a \p- b\a| \p< \epsilon$. \par
\a{C0)} By the absolute value definition, $\a|0\a| \p= 0$. \par

  \vs
  \hs \psc{We show} that \asc{if} $a$ {\bf equals} $b$, \asc{then} for every positive $\epsilon\p\in \R^+$ it's true that $\a|a \p- b\a| \p< \epsilon$. \par
    \hs\hs\a{H1)} \psc{Let} $a$ {\bf equal} $b$. \par
    \hs\hs\a{H2)} \psc{Let} $\epsilon \p\in \R^+$. \par
    \hs\hs \psc{We show} that $\a|a \p- b\a| \p< \epsilon$. \par
      \hs\hs\hs\psc{Since}, by \a{H1)}, $a \p= b$,        \psc{then}, by the existence of additive inverses for reals, \a{C1)} $a \p- b \p= 0$. \par
      \hs\hs\hs\psc{Since}, by \a{C1)}, $a \p- b \p = 0$, \psc{then}, by the fundamental meta-theorem of equations, \a{C2)} $\a|a \p- b\a| \p= \a|0\a|$. \par
      \hs\hs\hs\psc{Since}, by \a{C2)}, $\a|a \p- b\a| \p= \a|0\a|$, \psc{and}, by \a{C0)} $\p|0\p|\p = 0$, \psc{then}, by replacement, \a{C3)} $\a|a \p- b\a| \p= 0$. \par

      \hs\hs\hs\psc{Since}, by the $\R$ axioms, $0$ is under every positive real,
      \psc{and},  by \a{H2)}, $\epsilon$ is positive,
      \psc{then}, by replacement, \a{C4)} $0$ is under $\epsilon$. \par

      \hs\hs\hs\psc{Since}, by \a{C3)}, $\a|a \p- b\a| \p=  0$,
      \psc{and},  by \a{C4)}, $0 \p< \epsilon$,
      \psc{then}, by replacement, $\a|a \p- b\a| \p<  \epsilon$. \par

    \hs\hs \psc{This shows} that $\a|a \p- b\a| \p< \epsilon$. \par
  \hs\psc{This shows} that \a{C5)} \asc{if} $a$ {\bf equals} $b$, \asc{then} for every positive $\epsilon\p\in \R^+$ it's true that $\a|a \p- b\a| \p< \epsilon$.

  \vs
  \hs \psc{We show} that \asc{if} for every positive $\epsilon\p\in \R^+$ it's true that $\a|a \p- b\a| \p< \epsilon$, \asc{then} $a$ {\bf equals} $b$. \par
    \hs\hs\a{H3)} \psc{Let} $\epsilon \p\in \R^+$. \par
    \hs\hs\a{H4)} \psc{Let} $\a|a \p- b\a| \p< \epsilon$. \par
    \hs\hs\a{H5)} \psc{Let} $a$ {\bf not equal} $b$, for \psc{contradiction}. \par
    \hs\hs We must find a contradiction. \par
      \hs\hs\hs\psc{Since}, by \a{H5)},  $a \p{\not=} b$, \psc{then}, by the existence of additive inverses for reals, \a{C6)} $a \p- b \p{\not=} 0$. \par
      \hs\hs\hs\psc{Since}, by \a{C6)},  $a \p- b \p{\not=} 0$, \psc{then}, by the fundamental meta-theorem of equations, \a{C7)} $\a|a \p- b\a| \p{\not=} \a|0\a|$. \par
      \hs\hs\hs\psc{Since}, by \a{C7)},  $\a|a \p- b\a| \p{\not=} \a|0\a|$, \psc{and}, by \a{C0)}, $\p|0\p|\p = 0$, \psc{then}, by replacement, \a{C8)} $\a|a \p- b\a| \p{\not=} 0$. \par
      \hs\hs\hs\psc{Since}, by \a{C8)},  $\a|a \p- b\a| \p{\not=} 0$, \psc{then}, by the trichomotoy of reals, \a{C9)} $\a|a \p- b\a| \p< 0$ or $\a|a \p- b\a| \p> 0$. \par
      \hs\hs\hs\psc{Since}, by \a{C9)}, $\a|a \p- b\a| \p< 0$ or $\a|a \p- b\a| \p> 0$, \psc{and} absolute values are always nonnegative, \psc{then} by $\p\lor$-elimination, \a{C10)} $\a|a \p- b\a| \p> 0$. \par
      \hs\hs\hs\psc{Since}, by \a{C10)}, $\a|a \p- b\a| \p> 0$, \psc{then}, by the positive reals definition $\R^+$, \a{C11)} $\a|a \p- b\a| \p\in \R^+$. \par
      \hs\hs\hs\psc{Since}, by \a{H3)} and \a{H4)}, for every $\epsilon \p\in \R^+$ it's true that $\a|a \p- b\a| \p< \epsilon$, \psc{then}, by the previous lemma, \a{C12)} $\a|a \p- b\a| \p\notin \R^+$. \par
      \hs\hs\hs\psc{Since}, by \a{C11)}, $\a|a \p- b\a| \p\in \R^+$, \psc{and}, by \a{C12)}, $\a|a \p- b\a| \p\notin \R^+$, \psc{then} there's a \psc{contradiction}. \par
    \hs\hs \psc{This shows} that $a$ {\bf equals} $b$, by the law of non-contradiction. \par
  \hs\psc{This shows} that \a{C13)} \asc{if} for every positive $\epsilon\p\in \R^+$ it's true that $\a|a \p- b\a| \p< \epsilon$, \asc{then} $a$ {\bf equals} $b$. \par

  \vs
  \hs\psc{Since}, by \a{C5)}, \asc{if} $a$ {\bf equals} $b$, \asc{then} for every positive $\epsilon\p\in \R^+$ it's true that $\a|a \p- b\a| \p< \epsilon$, \par
  \hs\psc{and},   by \a{C13)}, \asc{if} for every positive $\epsilon\p\in \R^+$ it's true that $\a|a \p- b\a| \p< \epsilon$, \asc{then} $a$ {\bf equals} $b$, \par
  \hs\psc{then},  by the \asc{iff} definition, $a$ {\bf equals} $b$ \asc{iff} for every positive $\epsilon\p\in \R^+$ it's true that $\a|a \p- b\a| \p< \epsilon$. \par

\vs
\psc{This shows} that $a$ {\bf equals} $b$ \asc{iff} for every positive $\epsilon\p\in \R^+$ it's true that $\a|a \p- b\a| \p< \epsilon$.

% ------------------------------------------------------------------------------------------------------------------------------
\vs
\theorem The triangle inequality for $\R$. \par
Let $a,b \p\in \R$ be real numbers. \par
  \hs{\bf 0)} $\a|a \p+ b\a|$ is at most $\a|a\a| \p+ \a|b\a|$. \par
In symbols, \par

  \hs for every $a,b \p\in \R \a\langle$ \par
  \hs\hs $\a|a \p+ b\a| ~\p\leq~ \a|a\a| \p+ \a|b\a|$ \par
  \hs $\a\rangle$. \par

\proof TODO




% ------------------------------------------------------------------------------------------------------------------------------
% ------------------------------------------------------------------------------------------------------------------------------
% ------------------------------------------------------------------------------------------------------------------------------
% ------------------------------------------------------------------------------------------------------------------------------
\chapter{The three fundamental theorems of calculus}

% ------------------------------------------------------------------------------------------------------------------------------
\vs\hrule\vskip1pt
\theorem The {\bf first fundamental lemma of calculus}, aka {\bf the mean value theorem for derivatives}, aka the {\bf local-to-global principle} of differential calculus. \par

% ------------------------------------------------------------------------------------------------------------------------------
\vs\hrule\vskip1pt
\theorem The {\bf second fundamental lemma of calculus}, aka {\bf the mean value theorem for integrals}, aka the {\bf local-to-global principle} of integral calculus. \par

% ------------------------------------------------------------------------------------------------------------------------------
\vs\hrule\vskip1pt
\theorem The {\bf first fundamental theorem of calculus}, aka the differential of the {\bf area function of a function} is the differential of the {\bf function}. \par

% ------------------------------------------------------------------------------------------------------------------------------
\vs\hrule\vskip1pt
\theorem The {\bf second fundamental theorem of calculus}, (high-dimensional) integration on a (high-dimensional) interior is (low-dimensional) integration on a (low-dimensional) boundary. \par

% ------------------------------------------------------------------------------------------------------------------------------
\vs\hrule\vskip1pt
\theorem The {\bf third fundamental theorem of calculus}, aka {\bf Taylor's differential expansion}, aka Taylor's analytic approximation, aka Taylor's theorem. \par




% ------------------------------------------------------------------------------------------------------------------------------
% ------------------------------------------------------------------------------------------------------------------------------
% ------------------------------------------------------------------------------------------------------------------------------
% ------------------------------------------------------------------------------------------------------------------------------
\chapter{The Riemann integral}

% ------------------------------------------------------------------------------------------------------------------------------
\vs
By the {\bf First Fundamental Theorem of Calculus}, if a function is {\bf Riemann integrable} and {\bf continuous},
then it has an {\bf antiderivative}. Also, the antiderivative is {\bf continuous}.

\vs
More specifically, by the First Fundamental Theorem of Calculus,
if a function $f$ is Riemann integrable and continuous,
then it has an antiderivative $F$, and the antiderative is precisely the (continuous) function $F: x \mapsto \int_{\a[a..x\a]} f$.




% ------------------------------------------------------------------------------------------------------------------------------
% ------------------------------------------------------------------------------------------------------------------------------
% ------------------------------------------------------------------------------------------------------------------------------
% ------------------------------------------------------------------------------------------------------------------------------
\chapter{Topology}

\vs
Topology is the study of \abf{continuous functions}. \par
To talk about continuous functions, we must talk about \abf{open sets}. \par
Open sets are {\it not} defined {\it directly}, but indirectly in terms of their set-theoretic {\it behavior}: how they behave under \abf{unions} and \abf{intersections}. \par
So, I can never tell you what an open set {\it is}, only {\it how it behaves}. It's its {\it behavior} that defines it. \par

% ------------------------------------------------------------------------------------------------------------------------------
\vs
\theorem The fundamental duality of {\bf open topologies} and {\bf closed topologies}. \par

% ------------------------------------------------------------------------------------------------------------------------------
\vs\hrule\vskip1pt
\lemma The fundamental lemma of continuity and compacteness. \par
Images of continuous functions on compact sets are compact. \par
If the domain of a continuous function is compact, then its image is compact. \par

% ------------------------------------------------------------------------------------------------------------------------------
\vs
\lemma \par
Let $X$ be a totally-ordered topological space. \par
  \hs{\bf 0)} \asc{if} $X$ has no min,  \asc{then} the 2-set of $\infty$-balls $\a\{B \p\subseteq X \pipe \a\exists a \p\in X \a\langle~ B \p= \a(a         \a{..} \p+\infty\a) ~\a\rangle \a\}$ is an open cover of $X$. \par
  \hs{\bf 1)} \asc{if} $X$ has no max,  \asc{then} the 2-set of $\infty$-balls $\a\{B \p\subseteq X \pipe \a\exists a \p\in X \a\langle~ B \p= \a(\p-\infty \a{..}         a\a) ~\a\rangle \a\}$ is an open cover of $X$. \par
  \hs{\bf 2)} \asc{if} $X$ has min $m$, \asc{then} the 2-set of $\infty$-balls $\a\{B \p\subseteq X \pipe \a\exists a \p\in X \a\langle~ B \p= \a(a         \a{..} \p+\infty\a) ~\a\rangle \a\}$ is an open cover of $X \p- \a\{m\a\}$. \par
  \hs{\bf 3)} \asc{if} $X$ has max $M$, \asc{then} the 2-set of $\infty$-balls $\a\{B \p\subseteq X \pipe \a\exists a \p\in X \a\langle~ B \p= \a(\p-\infty \a{..}         a\a) ~\a\rangle \a\}$ is an open cover of $X \p- \a\{M\a\}$. \par

\asc{proof} of {\bf 1)}. \par
\psc{Let} $X$ be a totally-ordered topological space. \par
\psc{Let} $X$ have no max. \par
\psc{Let} ${\cal B}$ be the 2-set of $\infty$-balls $\a\{B \p\subseteq X \pipe \a\exists a \p\in X \a\langle~ B \p= \a(\p-\infty \a{..} a\a) ~\a\rangle \a\}$. \par

\psc{We show} that ${\cal B}$ is an open cover of $X$. \par
\psc{Since} \psc{We show} that ${\cal B}$ is an open cover of $X$, \psc{then}, by the open cover definition, \psc{We show} that $X$ is a subset of $\p\cup{\cal B}$. \par

  \hs \psc{We show} that $x$ is an element of $\p\cup{\cal B}$. \par
    \hs\hs \psc{Let} $x$ {\bf not} be an element of $\p\cup{\cal B}$, for \psc{contradiction}. \par
    \hs\hs \psc{Since} $x$ is {\bf not} in $\p\cup{\cal B}$, \psc{then}, by negating the union definition, there doesn't exist $B \p\in {\cal B}$ so that $x \p\in B$. \par
    \hs\hs \psc{Since} $\neg\a\exists B \p\in \cal B \a\langle$ $x \p\in B$ $\a\rangle$, \psc{then}, by the rules of classical logic, $\a\forall B \p\in {\cal B} \a\langle$ $x \p\notin B$ $\a\rangle$. \par

    \vs
    \hs\hs \psc{Since} $X$ has no max, \psc{then}, by negating the max definition, there doesn't exist $M \p\in X$ so that for all $y \p\in X$ it's true that $y \p\leq M$. \par
    \hs\hs \psc{Since} $\neg\a\exists M \p\in X$  $\a\forall y \p\in X \a\langle$ $y \p\leq M$ $\a\rangle$, \psc{then}, by the rules of classical logic, $\a\forall M \p\in X$ $\a\exists y \p\in X \a\langle$ $y \p> M$ $\a\rangle$. \par
    \hs\hs \psc{Since} $\a\forall M \p\in X$ $\a\exists y \p\in X \a\langle$ $y \p> M$ $\a\rangle$, \psc{and} $x \p\in X$, \psc{then}, by plugging $M \p{:=} x$, there exists $x' \p\in X$ so that $x' \p> x$. \par
    \hs\hs \psc{Since} $x \p< x'$, \psc{and} $x \p\in X$, \psc{and} $x' \p\in X$, \psc{then}, by the ball definition, $x$ is in the ball $\a(\p-\infty \a{..} x'\a)$. \par
    \hs\hs \psc{Since} $x' \p\in X$, \psc{then}, by the ${\cal B}$ definition, the ball $\a(\p-\infty \a{..} x'\a)$ is in ${\cal B}$. \par
    \hs\hs \psc{Since} $\a(\p-\infty \a{..} x'\a) \p\in {\cal B}$, \psc{and} $x \p\in \a(\p-\infty \a{..} x'\a)$, \psc{then} there exists $B \p\in {\cal B}$ so that $x \p\in B$. \par

    \vs
    \hs\hs \psc{Since} $\a\forall B \p\in {\cal B} \a\langle$ $x \p\notin B$ $\a\rangle$, \psc{and} $\a\exists B \p\in {\cal B} \a\langle$ $x \p\in B$ $\a\rangle$, \psc{then} there's a \psc{contradiction}.

  \hs \psc{This shows} that $x$ is an element of $\p\cup{\cal B}$, by the law of non-contradiction. \par

\psc{This shows} that $X$ is a subset of $\p\cup{\cal B}$. \par
\psc{This shows} that $\p\cup{\cal B}$ is an open cover of $X$. \par

% ------------------------------------------------------------------------------------------------------------------------------
\vs
\theorem The extreme value theorem for topological spaces. \par
Let $X$ be a {\bf compact}         topological space. \par
Let $Y$ be a {\bf totally-ordered} topological space. \par
Let $f: X \to Y$ be continuous. \par
  \hs{\bf 0)} There exist $a,b \p\in X$ so that for every $x \p\in X$ it's true that $f[x] \p\in \a[f[a] \a{..} f[b]\a]$. \par
The point $f[a] \p\in X$ is called the \bbf{min}    of $f$. \par
The point $f[b] \p\in X$ is called the \bbf{max}    of $f$. \par
The point $a    \p\in X$ is called the \bbf{argmin} of $f$. \par
The point $b    \p\in X$ is called the \bbf{argmax} of $f$. \par

\proof \par
\psc{Let} $X$ be a {\bf compact}         topological space. \par
\psc{Let} $Y$ be a {\bf totally-ordered} topological space. \par
\psc{Let} $f: X \to Y$ be continuous. \par
\psc{We show} that there exist $a,b \p\in X$ so that for every $x \p\in X$ it's true that $f[x] \p\in \a[f[a] \a{..} f[b]\a]$. \par

\vs
\psc{Since} $X$ is compact \psc{and} $f$ is continuous,
\psc{then}, by the fundamental lemma of continuity and compactness, the image $f_*[X]$ is compact.

\vs
\psc{Let} $m$ be the min of $f_*[X]$. (Why does this exist? This is what we want to proof!) \par
\psc{Let} $M$ be the max of $f_*[X]$. (Why does this exist? This is what we want to proof!) \par
\psc{Since} $m$ is the min of $f_*[X]$, \psc{then}, by the min definition, $m$ is in $f_*[X]$. \par
\psc{Since} $M$ is the max of $f_*[X]$, \psc{then}, by the max definition, $M$ is in $f_*[X]$. \par
\psc{Since} $m \p\in f_*[X]$, \psc{then}, by the $f_*[X]$ definition, there exists $a \p\in X$ so that $f: a \mapsto m$. \par
\psc{Since} $M \p\in f_*[X]$, \psc{then}, by the $f_*[X]$ definition, there exists $a \p\in X$ so that $f: b \mapsto M$. \par

\vs
\psc{Let} $f_*[X]$ have no max, for \psc{contradiction}. \par
\psc{Let} ${\cal B}$ be the 2-set of $\infty$-balls $\a\{B \p\subseteq f_*[X] \pipe \a\exists y \p\in f_*[X] \a\langle~ B \p= \a(\p-\infty \a{..} y\a) ~\a\rangle \a\}$. \par
\psc{Since} the domain of $X$, \psc{and} the codomain of $f$ is $Y$, \psc{then} by the image definition, the image $f_*[X]$ is a subset of $Y$. \par
\psc{Since} $f_*[X]$ is a subset of $Y$, and $Y$ is totally-ordered, \psc{then}, by XX, $f_*[X]$ is totally ordered. \par
\psc{Since} $f_*[X]$ has no max, \psc{and} $f_*[X]$ is totally-ordered,
\psc{then}, by lemma XX, the 2-set ${\cal B}$ is an open cover of $f_*[X]$. \par
\psc{Since} the 2-set ${\cal B}$ is an open cover of $f_*[X]$, \psc{and} $f_*[X]$ is compact, \par
\psc{then}, by the compactness definition, it has a finite subcover $\a\{\a(\p-\infty\a{..}y_0\a), \a(\p-\infty\a{..}y_1\a), \ldots, \a(\p-\infty\a{..}y_n\a)\a\}$. \par
\psc{Since} the cover $\a\{\a(\p-\infty\a{..}y_0\a), \a(\p-\infty\a{..}y_1\a), \ldots, \a(\p-\infty\a{..}y_n\a)\a\}$ is finite, \psc{then} the set $\a\{y_0, y_1, \ldots, y_n\a\}$ of boundary points is finite. \par
\psc{Since} the set $\a\{y_0, y_1, \ldots, y_n\a\}$ is finite, \psc{then}, by XX, it has a maximum $M$. \par
\psc{Since} $M$ is the max of $\a\{y_0, y_1, \ldots, y_n\a\}$, \psc{then}, by the max definition, $M$ is an element of $\a\{y_0, y_1, \ldots, y_n\a\}$. \par
\psc{Since} $M$ is an element of $\a\{y_0, y_1, \ldots, y_n\a\}$, \psc{and} $\a\{y_0, y_1, \ldots, y_n\a\}$ is a subset of $f_*[X]$, \par
\psc{then} by the properties of subsets, $M$ is an element of $f_*[X]$. \par
\psc{Since} $M$ is an element of $f_*[X]$, \psc{and} $\a\{\a(\p-\infty\a{..}y_0\a), \a(\p-\infty\a{..}y_1\a), \ldots, \a(\p-\infty\a{..}y_n\a)\a\}$ covers $f_*[X]$, \par
\psc{then}, by the cover definition, $M$ is an element of the union $\p\cup \a\{\a(\p-\infty\a{..}y_0\a), \a(\p-\infty\a{..}y_1\a), \ldots, \a(\p-\infty\a{..}y_n\a)\a\}$. \par
\psc{Since} $M$ is an element of the union $\p\cup \a\{\a(\p-\infty\a{..}y_0\a), \a(\p-\infty\a{..}y_1\a), \ldots, \a(\p-\infty\a{..}y_n\a)\a\}$, \par
\psc{then}, by the union definition, there exists $\a(\p-\infty \a{..} y_i\a) \p\in \a\{\a(\p-\infty\a{..}y_0\a), \a(\p-\infty\a{..}y_1\a), \ldots, \a(\p-\infty\a{..}y_n\a)\a\}$ so that $M \p\in \a(\p-\infty \a{..} y_i\a)$. \par
\psc{Since} $M$ is an element of $\a(\p-\infty \a{..} y_i\a)$, \psc{and} $\a(\p-\infty \a{..} y_i\a)$ in an element of $\a\{\a(\p-\infty\a{..}y_0\a), \a(\p-\infty\a{..}y_1\a), \ldots, \a(\p-\infty\a{..}y_n\a)\a\}$, \par
\psc{then} $M \p\in \a(\p-\infty\a{..}y_0\a)$ or $M \p\in \a(\p-\infty\a{..}y_1\a)$ or $\ldots$ $M \p\in \a(\p-\infty\a{..}y_n\a)$. \par

\vs
\psc{Since} $M$ is an element of $\a\{y_0, y_1, \ldots, y_n\a\}$, \par
\psc{and} every element of $\a\{y_0, y_1, \ldots, y_n\a\}$ is a boundary point of an element of $\a\{\a(\p-\infty\a{..}y_0\a), \a(\p-\infty\a{..}y_1\a), \ldots, \a(\p-\infty\a{..}y_n\a)\a\}$, \par
\psc{then} $M$ is a boundary point of an element of $\a\{\a(\p-\infty\a{..}y_0\a), \a(\p-\infty\a{..}y_1\a), \ldots, \a(\p-\infty\a{..}y_n\a)\a\}$. \par
\psc{Since} $M$ is a boundary point of an element of $\a\{\a(\p-\infty\a{..}y_0\a), \a(\p-\infty\a{..}y_1\a), \ldots, \a(\p-\infty\a{..}y_n\a)\a\}$, \par
\psc{and} every element of $\a\{\a(\p-\infty\a{..}y_0\a), \a(\p-\infty\a{..}y_1\a), \ldots, \a(\p-\infty\a{..}y_n\a)\a\}$ is an open ball, \par
\psc{and} open balls don't contain boundary points, \par
\psc{then} $M \p\notin \a(\p-\infty\a{..}y_0\a)$ or $M \p\notin \a(\p-\infty\a{..}y_1\a)$ or $\ldots$ $M \p\notin \a(\p-\infty\a{..}y_n\a)$. \par

\vs
\psc{Since} $M \p\notin \a(\p-\infty\a{..}y_0\a)$ or $M \p\notin \a(\p-\infty\a{..}y_1\a)$ or $\ldots$ $M \p\notin \a(\p-\infty\a{..}y_n\a)$, \par
\psc{and}   $M \p\in \a(\p-\infty\a{..}y_0\a)$ or $M \p\in \a(\p-\infty\a{..}y_1\a)$ or $\ldots$ $M \p\in \a(\p-\infty\a{..}y_n\a)$, \par
\psc{then}  there's a \psc{contradiction}. \par



% ------------------------------------------------------------------------------------------------------------------------------
% ------------------------------------------------------------------------------------------------------------------------------
% ------------------------------------------------------------------------------------------------------------------------------
\vs\hrule\vskip1pt
\subsection{\bf [...]}

[...]




% ------------------------------------------------------------------------------------------------------------------------------
% ------------------------------------------------------------------------------------------------------------------------------
% ------------------------------------------------------------------------------------------------------------------------------
% ------------------------------------------------------------------------------------------------------------------------------
\chapter{Category theory}

% ------------------------------------------------------------------------------------------------------------------------------
\vs
A \abf{category} is \abf{dots}, \abf{arrows} (between {\bf dots}), and \abf{gluing conditions} (between {\bf arrows}). \par
The {\bf dots} and {\bf arrows} {\it can} be explicitly visualized (they're concrete {\it things}). \par
The {\bf gluing conditions} {\it can't} be explicitly visualized (they're abstract {\it meta-things}, or something). \par

% ------------------------------------------------------------------------------------------------------------------------------
\vs\hrule\vskip1pt
\example {\bf The two-equals-one axiom}. \par
I used to think that the arrows \par
\halign{#\hfil & #\hfil & #\hfil & #\hfil \cr
  \hs $f$   & $: X$ & $\to$ & $Y$ \cr
  \hs $g$   & $: Y$ & $\to$ & $Z$ \cr
  \hs $h$   & $: X$ & $\to$ & $Z$ \cr
  \hs $1_X$ & $: X$ & $\to$ & $X$ \cr
  \hs $1_Y$ & $: Y$ & $\to$ & $Y$ \cr
  \hs $1_Z$ & $: Z$ & $\to$ & $Z$ \cr
}
formed a category. But they don't. {\bf Dots} and {\bf arrows} alone don't make a category. We need \abf{gluing conditions}, too. \par

\vs
Trick question: {\it how many} {\bf arrows} does this category have? \par
I used to think it had $6$: $f,g,h,1_X,1_Y,1_Z$. But it doesn't. \par
It has $7$ arrows: \psc{since} the target of $f$ \asc{equals} the source of $g$, \psc{then}, by the category axioms, there exists a arrow $gf$. \par
So our collection of arrows grows by $1$: $f,g,h,1_X,1_Y,1_Z,gf$. \par
Or does it? \par
Notice that the target of $1_X$ equals the source of $f$, so we also get the arrow $f1_X$. \par
For analogous reasons, we also get the arrows $1_Yf,g1_Y,1_Zg,h1_X,1_Zh$. \par
So our collection of arrows grows to: $f,g,h,1_X,1_Y,1_Z,gf,f1_X,1_Yf,g1_Y,1_Zg,h1_X,1_Zh$. \par
Or does it? \par
The collection of arrows $f,g,h,1_X,1_Y,1_Z,gf,f1_X,1_Yf,g1_Y,1_Zg,h1_X,1_Zh$ on its own doesn't form a category: it's missing {\bf gluing conditions}.
And we can't just go about choosing any old {\bf gluing conditions} that we please; nope. Our {\bf gluing conditions} must satisfy the {\bf category axioms}. The following set of gluing conditions does the trick: \par
\halign{# & # & # \cr
  \hs $gf$   & $\p=$ & $h$ \cr
  \hs $f1_X$ & $\p=$ & $f$ \cr
  \hs $1_Yf$ & $\p=$ & $f$ \cr
  \hs $g1_Y$ & $\p=$ & $g$ \cr
  \hs $1_Zg$ & $\p=$ & $g$ \cr
  \hs $h1_X$ & $\p=$ & $h$ \cr
  \hs $1_Zh$ & $\p=$ & $h$ \cr
}\par
Aha! So under these {\bf gluing conditions}, the arrow $gf$ ``{\it equals}'' the arrow $h$ (whatever ``{\it equals}'' means), and similarly for other arrows. \par
This means that our collection of $6\p+7$ arrows \par
  \hs $f,g,h,1_X,1_Y,1_Z,gf,f1_X,1_Yf,g1_Y,1_Zg,h1_X,1_Zh$ \par
``{\it collapses down}'' to the original $6$ arrows \par
  \hs $f,g,h,1_X,1_Y,1_Z$. \par

\vs
\abf{Objects} and \abf{morphisms} can be {\it visualized} as {\bf dots} and {\bf arrows}. \par
But how do we {\it visualize} the fact that (for instance) $gf \p= h$? \par
I don't know, and I suspect we can't (it's a {\it meta-thing}...), because $gf$ is the composition of $f$ with $g$ (so $gf$ is a path of length $2$), but $h$ is a single arrow (it's a path of length $1$)! \par
How {\it can} the {\it two} arrows $f$ and $g$ {\it equal} the {\it one} arrow $h$? I don't know. It's just an axiom for this category. And I don't know how to visualize it. But I think of it as the axiom $2 \p= 1$: {\bf two arrows equal one arrow}. \par

\vs
So, for this collection of arrows, under these gluing conditions, the arrows $f,g,h$ satisfy the $2 \p= 1$ axiom. (And other arrows do as well.) \par

% ------------------------------------------------------------------------------------------------------------------------------
\vs\hrule\vskip1pt
When thinking about {\bf categories}: \par
  \hs we {\it try} to ``forget'' about the {\it internal structure} of {\bf objects}, and think of objects as {\it structureless point-particles}, \par
  \hs we {\it try} to ``forget'' about the {\bf objects} altogether, and think only in terms of the {\bf arrows}. \par

\vs
\abf{Categories} are \abf{posets} in the next dimension. \par
\abf{$\infty$-groupoids} are \abf{sets} in the next dimension.

% ------------------------------------------------------------------------------------------------------------------------------
\vs
\definition {\bf Categories}. The category axioms. \par
A \abf{category} ${\cal C}$ satisfies the following sentences. \par
  \hs{\bf 0)} {\it Existence of arrows}: \par
    \hs\hs there exists a class ${\bf Hom}[{\cal C}]$ of ${\cal C}$-{\bf arrows}. \par

  \hs{\bf 1)} {\it Existence of source-arrows and target-arrows}: \par
    \hs\hs for every ${\cal C}$-arrow $f \p\in {\bf Hom}[{\cal C}] \a\langle$ \par
    \halign{#\hfil&#\hfil&#\hfil\cr
      \hs\hs$|$\hs there exists a ${\cal C}$-arrow ${\bf S}f \p\in {\bf Hom}[{\cal C}]$ & (aka the \abf{source-arrow} of $f$) so that $\a\langle$ ${\bf S}{\bf S}f \p= {\bf S}f$ & \asc{and} ${\bf T}{\bf S}f \p= {\bf S}f$ $\a\rangle$ \asc{and} \cr
      \hs\hs$|$\hs there exists a ${\cal C}$-arrow ${\bf T}f \p\in {\bf Hom}[{\cal C}]$ & (aka the \abf{target-arrow} of $f$) so that $\a\langle$ ${\bf S}{\bf T}f \p= {\bf T}f$ & \asc{and} ${\bf T}{\bf T}f \p= {\bf T}f$ $\a\rangle$ \cr
    }
    \hs\hs $\a\rangle$.

  \hs{\bf 2)} {\it Existence of identity-arrows}: \par
    \hs\hs for every ${\cal C}$-arrow $f \p\in {\bf Hom}[{\cal C}]$ \par
    \halign{#\hfil&#\hfil&#\hfil\cr
      \hs\hs$|$\hs there exists a ${\cal C}$-arrow $1_{{\bf S}f} \p\in {\bf Hom}[{\cal C}]$ & (aka the \abf{identity-arrow} of ${\bf S}f$) so that $\a\langle$ ${\bf S}1_{{\bf S}f} \p= {\bf S}f$ & \asc{and} ${\bf T}1_{{\bf S}f} \p= {\bf S}f$ $\a\rangle$ \asc{and} \cr
      \hs\hs$|$\hs there exists a ${\cal C}$-arrow $1_{{\bf T}f} \p\in {\bf Hom}[{\cal C}]$ & (aka the \abf{identity-arrow} of ${\bf T}f$) so that $\a\langle$ ${\bf S}1_{{\bf T}f} \p= {\bf T}f$ & \asc{and} ${\bf T}1_{{\bf T}f} \p= {\bf T}f$ $\a\rangle$ \cr
    }
    \hs\hs $\a\rangle$.

  \hs{\bf 3)} {\it Existence of composite-arrows}: \par  % \hs\hs For every $f,g \p\in {\bf Hom}[{\cal C}] \a\langle$ \par
    \hs\hs for every ${\cal C}$-arrow $f \p\in {\bf Hom}[{\cal C}]$ and \par
    \hs\hs for every ${\cal C}$-arrow $g \p\in {\bf Hom}[{\cal C}] \a\langle$ \par
    \hs\hs$|$\hs \asc{if} ${\bf T}f \p= {\bf S}g$, \par
    \hs\hs$|$\hs \asc{then} there exists a ${\cal C}$-arrow $gf \p\in {\bf Hom}[{\cal C}]$ (aka the \abf{composite-arrow} of $f$ {\bf with} $g$) so that $\a\langle$ \par
    \hs\hs$|$\hs$|$\hs ${\bf S}gf \p= {\bf S}f$ \asc{and} \par
    \hs\hs$|$\hs$|$\hs ${\bf T}gf \p= {\bf T}g$ \par
    \hs\hs$|$\hs $\a\rangle$ \par
    \hs\hs $\a\rangle$.

\vs
\proposition {\bf Identity}-arrows and {\bf source}-arrows are the same. {\bf Identity}-arrows and {\bf target}-arrows are the same. \par
Let ${\cal C}$ be a category. \par
Let $f \p\in {\bf Hom}[{\cal C}]$ be a ${\cal C}$-arrow. \par
  \hs{\bf 0)}    $1_{{\bf S}f} \p= {\bf S}f$. \par
  \hs{\bf 1)}    $1_{{\bf T}f} \p= {\bf T}f$. \par
  \hs{\bf 0$'$)} The identity-arrow of the source-arrow of $f$ is the source-arrow of $f$. \par
  \hs{\bf 1$'$)} The identity-arrow of the target-arrow of $f$ is the target-arrow of $f$. \par




% ------------------------------------------------------------------------------------------------------------------------------
% ------------------------------------------------------------------------------------------------------------------------------
% ------------------------------------------------------------------------------------------------------------------------------
% ------------------------------------------------------------------------------------------------------------------------------
\chapter{Sheaves}

% ------------------------------------------------------------------------------------------------------------------------------
\vs
\abf{Sheaves} keep track of {\bf local-to-global} relationships between data in a way that ensures local-to-global consistency. \par

\vs
The idea is that we have a bunch of open sets of $X$ stuffed into a topology $\tau_X \p\subseteq {\cal P}X$. \par
And we take an open set $U \p\subseteq X$. \par
And we take an open cover of $U$, say, the open cover $\a\{\g{U_0}, \b{U_1}\a\} \p\subseteq \tau_X$ made of {\it two} cover elements. Since $\a\{\g{U_0}, \b{U_1}\a\}$ covers $X$, then $\g{U_0} \p\cup\b{U_1} \p= U$. \par
On each cover element $U_i \p\in \a\{\g{U_0}, \b{U_1}\a\}$ there is a continuous map $f_i: U_i \to \R$. \par
Since there are two cover elements ($\g{U_0}$ and $\b{U_1}$), and on each cover element there's a continuous map, then we have {\it two} continuous maps: \par
  \hs0) a continuous map $\g{f_0}: \g{U_0} \to \R$ on $\g{U_0}$, and \par
  \hs1) a continuous map $\b{f_1}: \b{U_1} \to \R$ on $\b{U_1}$. \par
And we want to look at all possible intersections of all cover elements. \par
So, we take all four interections of $\g{U_0}$ and $\b{U_1}$: \par
  \hs0) $\g{U_0} \p\cap \g{U_0}$, which is just $\g{U_0}$, \par
  \hs1) $\g{U_0} \p\cap \b{U_1}$, \par
  \hs2) $\b{U_1} \p\cap \g{U_0}$, which is the same as $\g{U_0} \p\cap \b{U_1}$, \par
  \hs3) $\b{U_1} \p\cap \b{U_1}$, which is just $\b{U_1}$. \par
% (which is the same as the intersection $\b{U_1} \p\cap \g{U_0}$ of $\b{U_1}$ with $\g{U_0}$). \par
This yields {\it two extra} continuous maps: \par
  \hs0) the restriction of $\g{f_0}: \g{U_0} \to \R$ to $\g{U_0} \p\cap \b{U_1}$, which is denoted $\g{f_0}|_{\g{U_0} \p\cap \b{U_1}}: \g{U_0} \p\cap \b{U_1} \to \R$, and \par
  \hs1) the restriction of $\b{f_1}: \b{U_1} \to \R$ to $\g{U_0} \p\cap \b{U_1}$, which is denoted $\b{f_1}|_{\g{U_0} \p\cap \b{U_1}}: \g{U_0} \p\cap \b{U_1} \to \R$. \par
So, we started with two maps, $\g{f_0}$ and $\b{f_1}$, but now we have four: \par
  \hs0) $\g{f_0}: \g{U_0} \to \R$, \par
  \hs1) $\b{f_1}: \b{U_1} \to \R$, \par
  \hs2) $\g{f_0}|_{\g{U_0} \p\cap \b{U_1}}: \g{U_0} \p\cap \b{U_1} \to \R$, and \par
  \hs3) $\b{f_1}|_{\g{U_0} \p\cap \b{U_1}}: \g{U_0} \p\cap \b{U_1} \to \R$. \par
In general, $\g{f_0}: \g{U_0} \to \R$ and $\b{f_1}: \b{U_1} \to \R$ are completely different maps. \par
And, in general, their restrictions $\g{f_0}|_{\g{U_0} \p\cap \b{U_1}}: \g{U_0} \p\cap \b{U_1} \to \R$ and $\b{f_1}|_{\g{U_0} \p\cap \b{U_1}}: \g{U_0} \p\cap \b{U_1} \to \R$ are completely different maps. \par

\vs
Now comes the good stuff. \par
We want to ``glue'' $\g{f_0}$ and $\b{f_1}$, which are defined on $\g{U_0} \p\subseteq U$ and $\b{U_1} \p\subseteq U$, into a {\bf single map} $f$ defined on all of $\g{U_0} \p\cup \b{U_1}$ (which is $U$). \par
{\it But} there isn't a single map defined on all of $\g{U_0} \p\cup \b{U_1}$: there are {\bf two maps}! Call them $f: \g{U_0} \p\cup \b{U_1} \to \R$ and $g: \g{U_0} \p\cup \b{U_1} \to \R$. \par
The map $f: \g{U_0} \p\cup \b{U_1} \to \R$ is defined piecewise, as follows. \par
  \hs0) For every $x$, \asc{if} $x$ is in $\g{U_0} \p-    \b{U_1}$, \asc{then} $f$ maps $x$ to $\g{f_0}[x]$. \par
  \hs1) For every $x$, \asc{if} $x$ is in $\b{U_1} \p-    \g{U_0}$, \asc{then} $f$ maps $x$ to $\b{f_1}[x]$. \par
  \hs2) For every $x$, \asc{if} $x$ is in $\g{U_0} \p\cap \b{U_1}$, \asc{then} $f$ maps $x$ to $\g{f_0}|_{\g{U_0} \p\cap \b{U_1}}[x]$. \par
The map $g: \g{U_0} \p\cup \b{U_1} \to \R$ is defined piecewise, as follows. \par
  \hs0) For every $x$, \asc{if} $x$ is in $\g{U_0} \p-    \b{U_1}$, \asc{then} $g$ maps $x$ to $\g{f_0}[x]$. \par
  \hs1) For every $x$, \asc{if} $x$ is in $\b{U_1} \p-    \g{U_0}$, \asc{then} $g$ maps $x$ to $\b{f_1}[x]$. \par
  \hs2) For every $x$, \asc{if} $x$ is in $\g{U_0} \p\cap \b{U_1}$, \asc{then} $g$ maps $x$ to $\b{f_1}|_{\g{U_0} \p\cap \b{U_1}}[x]$. \par
By definition, the maps $f$ and $g$ agree on $\g{U_0} \p- \b{U_1}$ and on $\b{U_1} \p- \g{U_0}$, but they disagree on the intersection $\g{U_0} \p\cap \b{U_1}$, because $\g{f_0}|_{\g{U_0} \p\cap \b{U_1}}[x]$ need not equal $\b{f_1}|_{\g{U_0} \p\cap \b{U_1}}[x]$... \par
But we can {\it demand} that $f$ and $g$ agree $\g{U_0} \p\cap \b{U_1}$ too, and, in that case, $f$ and $g$ become the same map, ie. $f \p= g$. \par
So, if we want $f \p= g$ to be true, then we keep the piecewise definitions of $f$ and $g$, and we add an extra condition: \par
  \hs For every $x$, \asc{if} $x$ is in $\g{U_0} \p\cap \b{U_1}$, \asc{then} $\g{f_0}|_{\g{U_0} \p\cap \b{U_1}}[x] \p= \b{f_1}|_{\g{U_0} \p\cap \b{U_1}}[x]$. \par
This condition ensures that $f: \g{U_0} \p\cup \b{U_1} \to \R$ and $g: \g{U_0} \p\cup \b{U_1} \to \R$ are the same map, ie. $f \p= g$. \par

\vs
And now we have a single {\bf patchwerk map} $f: \g{U_0} \p\cup \b{U_1} \to \R$ defined on all of $\g{U_0} \p\cup \b{U_1}$, constructed by ``gluing'' $\g{f_0}: \g{U_0} \to \R$ and $\b{f_1}: \b{U_1} \to \R$. \par
Since $\g{f_0}: \g{U_0} \to \R$ and $\b{f_1}: \b{U_1} \to \R$ are continuous, then the patchwerk map $f: \g{U_0} \p\cup \b{U_1} \to \R$ is also continuous, but this requires proof. \par
{\bf Patchwerk} is a boss in {\it World of Warcraft}, made by stitching together corpses.

% ------------------------------------------------------------------------------------------------------------------------------
\vs\hrule\vskip1pt
\definition {\bf Presheaves} (of abelian groups) on topological spaces. \par
Let $(X, \tau_X)$ be a topological space. \par
Let {\bf Ab} be the category of abelian groups. \par
A \abf{presheaf} ${\cal F}$ (of abelian groups) on the topological space $(X, \tau_X)$ \asc{is} \par
  \hs a {\bf contravariant functor} ${\cal F}$ from $\tau_X$ to {\bf Ab}, or equivalently \par
  \hs a {\bf covariant functor} ${\cal F}$ from $\tau_X^{op}$ to {\bf Ab}. \par
In detail. \par

\hs{\bf 0)} For every $\tau_X$ arrow $f$ \par
\halign{#\hfil&#\hfil \cr
  \hs\hs\hs there exists an {\bf Ab} arrow ${\cal F}f$            & so that ${\bf S}{\cal F}f \p= {\cal F}{\bf S}f$ \asc{and} \cr
  \hs\hs\hs there exists an {\bf Ab} arrow ${\cal F}f$            & so that ${\bf T}{\cal F}f \p= {\cal F}{\bf T}f$ \asc{and} \cr
  \hs\hs\hs there exists an {\bf Ab} arrow ${\cal F}1_{{\bf S}f}$ & so that ${\cal F}1_{{\bf S}f} \p= 1_{{\cal F}{\bf S}f}$ \asc{and} \cr
  \hs\hs\hs there exists an {\bf Ab} arrow ${\cal F}1_{{\bf T}f}$ & so that ${\cal F}1_{{\bf T}f} \p= 1_{{\cal F}{\bf T}f}$. \cr
}

\hs{\bf 0)} {\it Existence of arrows}: \par
\hs\hs for every $\tau_X$ arrow $f:U \to V$ \par
\hs\hs\hs there exists an {\bf Ab} arrow ${\cal F}f: {\cal F}U \from {\cal F}V $. \par

\hs{\bf 1)} {\it Composition compatibility}: \par
\hs\hs for every $\tau_X$ arrow $f:U \to V$ and \par
\hs\hs for every $\tau_X$ arrow $g:V \to W$ \par
\hs\hs\hs there exists an {\bf Ab} arrow ${\cal F}gf: {\cal F}U \from {\cal F}W$ \par
\hs\hs\hs\hs so that ${\cal F}gf \p= {\cal F}f{\cal F}g$. \par

\hs{\bf 2)} {\it Object/identity compatibility}: \par
\hs\hs for every $\tau_X$ identity arrow $1_U:U \to U$ \par
\hs\hs\hs there exists an {\bf Ab} identity arrow ${\cal F}1_U: {\cal F}U \from {\cal F}U$ \par
\hs\hs\hs\hs so that ${\cal F}1_U \p= 1_{{\cal F}U}$. \par

\example \par
\psc{Let} $\tau_X$ be a category. \par
\psc{Let} {\bf Ab} be a category. \par
\psc{Let} $f: U \to V$ be a $\tau_X$ arrow. \par
\psc{Let} $g: V \to W$ be a $\tau_X$ arrow. \par
\psc{Let} ${\cal F}$ be an {\bf Ab}-presheaf on $\tau_X$. \par

$|$\hs \par
$|$\hs \psc{Since} $\tau_X$ is a category, \par
$|$\hs \psc{and}   $f: U \to V$ is a $\tau_X$ arrow from $U$ to $V$, \par
$|$\hs \psc{and}   $g: V \to W$ is a $\tau_X$ arrow from $V$ to $W$, \par
$|$\hs \psc{and}   ${\bf Tar}[f] \p= {\bf Src}[g]$, \par
$|$\hs \psc{then}, by the category axioms, there exists a $\tau_X$ arrow $gf: U \to W$ from $U$ to $W$. \par

$|$\hs \par
$|$\hs \psc{Since} $f:  U \to V$ is a $\tau_X$ arrow from $U$ to $V$, \par
$|$\hs \psc{and}   $g:  V \to W$ is a $\tau_X$ arrow from $V$ to $W$, \par
$|$\hs \psc{and}   $gf: U \to W$ is a $\tau_X$ arrow from $U$ to $W$, \par
$|$\hs \psc{and}   ${\cal F}$ is an {\bf Ab}-presheaf on $\tau_X$, \par
$|$\hs \psc{then}, by presheaf arrow compatibility, there exists an {\bf Ab} arrow ${\cal F}f:  {\cal F}U \from {\cal F}V$ to ${\cal F}U$ from ${\cal F}V$, \par
$|$\hs \psc{and},  by presheaf arrow compatibility, there exists an {\bf Ab} arrow ${\cal F}g:  {\cal F}V \from {\cal F}W$ to ${\cal F}V$ from ${\cal F}W$, \par
$|$\hs \psc{and},  by presheaf arrow compatibility, there exists an {\bf Ab} arrow ${\cal F}gf: {\cal F}U \from {\cal F}W$ to ${\cal F}U$ from ${\cal F}W$. \par

$|$\hs \par
$|$\hs \psc{Since} {\bf Ab} is a category, \par
$|$\hs \psc{and}   ${\cal F}f: {\cal F}U \from {\cal F}V$ is an {\bf Ab} arrow to ${\cal F}U$ from ${\cal F}V$, \par
$|$\hs \psc{and}   ${\cal F}g: {\cal F}V \from {\cal F}W$ is an {\bf Ab} arrow to ${\cal F}V$ from ${\cal F}W$, \par
$|$\hs \psc{and}   ${\bf Tar}[{\cal F}g] \p= {\bf Src}[{\cal F}f]$, \par
$|$\hs \psc{then}, by the category axioms, there exists an {\bf Ab} arrow ${\cal F}f{\cal F}g: {\cal F}U \from {\cal F}W$ to ${\cal F}U$ from ${\cal F}W$. \par

$|$\hs \par
$|$\hs By presheaf composition compatibility, ${\cal F}gf \p= {\cal F}f{\cal F}g$. \par




% ------------------------------------------------------------------------------------------------------------------------------
% ------------------------------------------------------------------------------------------------------------------------------
% ------------------------------------------------------------------------------------------------------------------------------
% ------------------------------------------------------------------------------------------------------------------------------
\end
